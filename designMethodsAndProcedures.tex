\documentclass[12pt,oneside,letterpaper,titlepage]{article}
\author{Tim Visher\\ CMSCU 498\\ Chestnut Hill College}
\title{Design: Methods and Procedures}

\usepackage[colorlinks=true]{hyperref}
\usepackage{paralist}
\usepackage{listings}
\lstset{basicstyle=\small}

\begin{document}

\maketitle

\tableofcontents

\newpage

\section{Introduction to Research Project}

\subsection{Background necessary to understand the research}

This section will outline the different areas of knowledge that a reader of this
paper will need to be familiar with on order to be able to successfully
interpret the research presented.

The reader should:

\begin{itemize}

\item be familiar with various aspects of hardware performance, especially those
  associated with inter-charge mobile life expectancy.  Much of the research
  here-in has to do with how computer usage and hardware design affects this
  life expectancy and what can be done to make mobile devices last longer on a
  single charge.

\item be familiar with Cloud Services and how they are used.  The research deals
  heavily with the different demands that Cloud Services make on mobile
  computing hardware compared to traditional desktop equivalents of the same
  services.

\item be familiar with the offerings available currently in today's mobile
  computing environment.  The research will discuss various performance costs
  and benefits in relation to traditional notebooks, netbooks, and smart-phones.

\item a cursory understanding of software testing.  Some of the procedures I
  will be using will center around automating the testing of various aspects of
  computer performance on various mobile platforms as well as in the cloud.

\end{itemize}

\subsection{Reiteration of problem statement, hypothesis or questions being addressed}

The problem that this research project is addressing is as follows: Currently,
adoption of ultra-mobile PCs has primarily been driven by price.  The author
believes that the reason wider spread adoption hasn't been seen yet is of a dual
nature:

\begin{enumerate}

\item Mobile platform providers are still designing their machines to primarily
  behave as a stripped down desktop.  As such, they do not optimize for use of
  Cloud Services in such a way that their device will experience longer battery
  life and better performance.

\item Consumers still believe that computers are primarily for executing tasks
  locally.  The author believes that this is no longer true, both from personal
  experience as well as from usage reports.  However, consumer perception is as
  if not more important than actual facts.

\end{enumerate}

The problem, then, is discovering what factors would have to coincide to produce
an ideal mobile platform for the public consumer.

The questions and hypotheses being addressed are as follows:

\begin{enumerate}

  \item The author's main hypothesis is that mass adoption of ultra-mobile PCs
    will not be seen until hardware offerings pass muster.

  Tightly integrated here are 2 questions:

  \begin{enumerate}

    \item What about current mobile platform offerings is unsatisfying to the
      average consumer?

    \item What would the ideal public consumer mobile platform, designed to take
      full advantage of cloud services, look like?

  \end{enumerate}

  \item A corollary hypothesis is that Netbooks and Notebooks perform nearly
    indistinguishably from each other when utilizing cloud services despite the
    observable differences in performance when using local applications.

  \item Another hypothesis that there likely will be no time to examine is that
    average consumer perception of computer use does not adequately reflect
    reality.

  \item A final, tangential question that the author is interested in is to what
    degree privacy concerns are holding back public adoption of cloud services.

\end{enumerate}

\subsection{Overview of Procedure}

The procedure that the researcher wishes to pursue for this project is split
into 2 areas of investigation: (1) hampered mobile platform adoption and; (2)
netbook/notebook performance comparability.

\subsubsection{Hampered Mobile Platform Adoption}

The author intends to investigate research that has already been done in the
area of battery life performance.  It is somewhat common knowledge that a bright
display, WiFi, and Bluetooth are the biggest killers of battery life.  The
author would like to investigate if there is verifiable truth to that claim.

\subsubsection{Netbook/Notebook Performance Comparability}

The primary strategy for investigating Netbook vs. Notebook performance will be
testing performance for services offered both locally and in the cloud.

This will attempt to verify that there is indeed little to no difference between
Netbook and Notebook performance when the cloud is what is being utilized.

\section{Subjects/Participants}

\subsection{Information necessary to replicate the study with respect to subjects}

The primary participants in this study will be Chestnut Hill College students.
This is primarily due to resource and time constraints.  However, the online
survey has the potential to reach a far wider audience.  Also, video interviews
could be conducted at other sites as well as at CHC.

\subsection{Information necessary to realize inability to generalize because of some special property of the subjects}

Because of the nature of the primary subjects, it would be quite difficult to
generalize based on the fact that they will mostly be college students and
faculty.  In order to truly generalize, a stratified and representative sampling
of the population would have to be made.  This is not within the resources of
the researcher to accomplish and therefore is a necessary shortcoming.

\section{Instrumentation}

%What do you need to do your project.  Only a couple paragraphs.

\subsection{Information necessary to exactly replicate the study with respect to apparatus or setting}

Equipment needed to reconduct my study would be
\begin{inparaenum}[(1)]
\item A Macbook;
\item An acer ASPIRE one;
\item Several Browsers;
\item OpenOffice.org;
\item The testing SpreadSheet.
\end{inparaenum}

\begin{enumerate}

\item A MacBook with the following specifications:

  \begin{tabular}{| r | p{5cm} |}
    \hline
    Model Identifier      & MacBook1,1 1 \\ \hline
    Processor Name        & Intel Core Duo \\ \hline
    Processor Speed       & 1.83 GHz \\ \hline
    Number Of Processors  & 1 \\ \hline
    Total Number Of Cores & 2 \\ \hline
    L2 Cache              & 2 MB \\ \hline
    Memory                & 2 GB \\ \hline
    Bus Speed             & 667 MHz \\ \hline
    Boot ROM Version      & MB11.0061.B03 \\ \hline
    SMC Version (system)  & 1.4f12 \\ \hline
    Mac OS X              & 10.6.3 \\ \hline
    Network  Adapter      & Marvell Yukon Gigabit Adapter 88E8053 Singleport
                            Copper SA \\
    \hline
  \end{tabular}

  with the following software installed at the noted versions:

  \begin{tabular}{| r | p{5cm} |}
    \hline
    Google Chrome   & 5.0.342.9 \\ \hline
    Safari          & 4.0.5 (6531.22.7) \\ \hline
    Mozilla Firefox & Mozilla/5.0 (Macintosh; U; Intel Mac OS X 10.6; en-US;
    rv:1.9.2) Gecko/20100115 Firefox/3.6 \\ \hline
    OpenOffice.org  & 3.2.0 OOO320m12 (Build:9483) \\
    \hline
  \end{tabular}

\item An acer ASPIRE one with the following specifications:

  \begin{tabular}{| r | p{5cm} |}
    \hline
    Model                        & Acer Aspire one V1.07 \\ \hline
    Enclosure Type               & Notebook \\ \hline
    Operating System             & Windows XP Home Edition Service Pack 3 (build
    2600) \\ \hline
    Processor Name               & Intel Atom N270 \\ \hline
    Processor Speed              & 1.60 GHz \\ \hline
    Number of Processors         & 1 \\ \hline
    Total Number of Cores        & 1 \\ \hline
    Primary Cache                & 48 KB \\ \hline
    Secondary Cache              & 512 KB \\ \hline
    Hyper-threading              & 2 total \\ \hline
    Main Memory                  & 1 GB \\ \hline
    Network Adapter              & Atheros AR5007EG Wireless Network Adapter \\
    \hline
  \end{tabular}

  with the following software installed at the noted versions:

  \begin{tabular}{| r | p{5cm} |}
    \hline
    Google Chrome                & 4.1.249.1045 (42898) \\ \hline
    Safari                       & 4.0.5 (531.22.7) \\ \hline
    Mozilla Firefox              & Mozilla/5.0 (Window; U; Windows NT 5.1;
                                   en-US; rv:1.9.2.3) Gecko/20100401
                                   Firefox/3.6.3 (.NET CLR 3.5.30729) \\ \hline
    OpenOffice.org               & 3.2.0 OOO320m12 (Build:9483) \\
    \hline
  \end{tabular}

% \item Online resources to conduct online research

\item A Google account with access to Google Docs for spreadsheet analysis.

\item The SpreadSheet.\footnote{SpreadSheet published here:
  \url{https://spreadsheets.google.com/ccc?key=0Ar8ZDnsVfQzLdDh2MjhfU1JoOExLejlMTHpjY1NrdWc&hl=en}}

\item A clock or stopwatch to measure performance.

\end{enumerate}

\subsection{Information necessary to realize inability to generalize because of some special property of the apparatus or the setting}

It is impossible to generalize this research because there was not enough time
or resources to truly test on varied hardware and over various connection.
Also, performance testing was somewhat arbitrary as actual sampling could not be
accomplished.  Instead, observation alone was used in conjunction with the
system monitoring applications of the respective operating systems.

\section{Procedure/Methodology}

%Expected recipe

\subsection{Specific procedure used to answer the problem}

The researcher used the following Groovy script to generate the random numbers
used to seed the functions of the spreadsheet:

\lstinputlisting[frame=single,breaklines]{../randomSourceGen.groovy}
\section{Delimitations}

%Suggestions for future research?

Due to time constraints, the author cannot investigate the degree to which
privacy concerns are affecting mobile platform adoption.  Certainly, concepts of
privacy are changing rapidly, but to what degree are people comfortable truly
out sourcing much of their private data is at this point, unknown.  Privacy
concerns may be a critical factor in mobile platform adoption and if that is the
case, designing the perfect mobile platform would have little or no effect on
adoption.

\section{Results}

\subsection{In general, what was found? What happened?}

\subsection{Data provided to justify statements or major trends?}

\subsection{Reliability of the results?}

\subsection{How was reliability demonstrated?}

\end{document}
