\documentclass[12pt,oneside,letterpaper,titlepage]{article}
\author{Tim Visher\\ CMSCU 498\\ Chestnut Hill College}
\title{Design: Methods and Procedures}

\usepackage[colorlinks=true]{hyperref}

\begin{document}

\maketitle

\tableofcontents

\newpage

\section{Introduction to Research Project}

\subsection{Background necessary to understand the research}

This section will outline the different areas of knowledge that a reader of this
paper will need to be familiar with on order to be able to successfully
interpret the research presented.

The reader should:

\begin{itemize}

\item be familiar with various aspects of hardware performance, especially those
  associated with inter-charge mobile life expectancy.  Much of the research
  here-in has to do with how computer usage and hardware design affects this
  life expectancy and what can be done to make mobile devices last longer on a
  single charge.

\item be familiar with Cloud Services and how they are used.  The research deals
  heavily with the different demands that Cloud Services make on mobile
  computing hardware compared to traditional desktop equivalents of the same
  services.

\item be familiar with the offerings available currently in today's mobile
  computing environment.  The research will discuss various performance costs
  and benefits in relation to traditional notebooks, netbooks, and smart-phones.

\item a cursory understanding of software testing.  Some of the procedures I
  will be using will center around automating the testing of various aspects of
  computer performance on various mobile platforms as well as in the cloud.

\end{itemize}

\subsection{Reiteration of problem statement, hypothesis or questions being addressed}

The problem that this research project is addressing is as follows: Currently,
adoption of ultra-mobile PCs has primarily been driven by price.  The author
believes that the reason wider spread adoption hasn't been seen yet is of a dual
nature:

\begin{enumerate}

\item Mobile platform providers are still designing their machines to primarily
  behave as a stripped down desktop.  As such, they do not optimize for use of
  Cloud Services in such a way that their device will experience longer battery
  life and better performance.

\item Consumers still believe that computers are primarily for executing tasks
  locally.  The author believes that this is no longer true, both from personal
  experience as well as from usage reports.  However, consumer perception is as
  if not more important than actual facts.

\end{enumerate}

The problem, then, is discovering what factors would have to coincide to produce
an ideal mobile platform for the public consumer.

The questions and hypotheses being addressed are as follows:

\begin{enumerate}

  \item The author's main hypothesis is that mass adoption of ultra-mobile PCs
    will not be seen until hardware offerings pass muster.

  Tightly integrated here are 2 questions:

  \begin{enumerate}

    \item What about current mobile platform offerings is unsatisfying to the
      average consumer?

    \item What would the ideal public consumer mobile platform, designed to take
      full advantage of cloud services, look like?

  \end{enumerate}

  \item A corollary hypothesis is that Netbooks and Notebooks perform nearly
    indistinguishably from each other when utilizing cloud services despite the
    observable differences in performance when using local appications.

  \item Another hypothesis that there likely will be no time to examine is that
    average consumer perception of computer use does not adequately reflect
    reality.

  \item A final tongential question that the author is interested in is to what
    degree privacy concerns are holding back public adoption of cloud services.

\end{enumerate}

\subsection{Overview of Procedure}

%Case for why the procedure is appropriate and the best procedure given the
%situtation.

The procedure that the researcher wishes to pursue for this project is split
into 3 strategies across the 2 main areas of investigation. %TODO can these be
                                                            %automatically
                                                            %itemized?

\subsubsection{Hampered Mobile Platform Adoption}

\begin{enumerate}

\item The primary strategy for researching this set of hypotheses and questions
                                               %TODO Can I Hyperlink
                                               %intentionally to these
                                               %questions?
  will be consumer surveying and interviewing.  The author intends to:

  \begin{enumerate}

  \item Hold video interviews with CHC college students and faculty to ask them
    what about todays mobile offerings doesn't satisfy them, and what they think
    would push them over the edge of adoption.

  \item Construct and administer an online survey that can be opened to the
    public in order to get a more stratified view of the issues hampering
    adoption.

  \end{enumerate}

  This strategy attempts to address the lack of knowledge about what consumers
  are actually looking for in their mobile hardware.

\item Secondarily, the author intends to investigate research that has already
  been done in the area of battery life performance.  It is somewhat common
  knowledge that a bright display, WiFi, and Bluetooth are the biggest killers
  of battery life.  The author would like to investigate if there is verifiable
  truth to that claim.

\end{enumerate}

\subsubsection{Netbook/Notebook Performance Comparability}

\begin{enumerate}

\item The primary strategy for investigating Netbook vs. Notebook performance
  will be testing performance for services offered both locally and in the
  cloud.

  This will attempt to verify that there is indeed little to no difference
  between Netbook and Notebook performance when the cloud is what is being
  utilized.

\item Secondarily, the author intends to conduct blind testing in which a
  representative sampling of users utilize different cloud services and attempts
  to identify which computer is a Netbook and which is a Notebook.

  This strategy attempts to investigate the question of whether or not usage
  experience is truly different between traditional mobile platforms and todays
  mobile offerings while removing any possible bias against Netbooks for simply
  being Netbooks.

\end{enumerate}

\section{Subjects/Participants}

\subsection{Information necessary to replicate the study with respect to subjects}

The primary participants in this study will be Chestnut Hill College students.
This is primarily due to resource and time constraints.  However, the online
survey has the potential to reach a far wider audience.  Also, video interviews
could be conducted at other sites as well as at CHC.

\subsection{Information necessary to realize inability to generalize because of some special property of the subjects}

Because of the nature of the primary subjects, it would be quite difficult to
generalize based on the fact that they will mostly be college students and
faculty.  In order to truly generalize, a stratified and representative sampling
of the population would have to be made.  This is not within the resources of
the researcher to accomplish and therefore is a necessary shortcoming.

\section{Instrumentation}

%What do you need to do your project.  Only a couple paragraphs.

\subsection{Information necessary to exactly replicate the study with respect to apparutus or setting}

Equipment needed to reconduct my study would be:

\begin{enumerate}

\item A Netbook

\item A Notebook

\item Recording equipment to conduct interviews

\item Online resources to conduct online research

\item Computer hardware and testing software to test performance locally and via
  the cloud.

\end{enumerate}

\subsection{Information necessary to realize inability to generalize beacuse of some special property of the apparatus or the setting}

Exact reproduction of this study would require access to Chestnut Hill College
which is not always possible.  It may not be possible to completely reproduce
because of that.  Also, the times in which the research is being conducted are
changing quickly and the cultural mood and attitude will have shifted by the
time this research is published.  This is another reason the research may not be
possible to generalize. 

\section{Procedure/Methodology}

%Expected recipe

\subsection{Specific procedure used to answer the problem}

\subsection{Information necessary to exactly replicate the study with respect to the procedure}

\subsection{Procedures used to deal with unavoidable problems}

\subsection{Information necessary to realize inability to generalize because of some special property of the procedure}

\subsection{Was a baseline obtained or were the groups equal at the start?}

\subsection{How were the independent variables measured?}

\subsection{How were the dependent variables measured}

\section{Delimitations}

%Suggestions for future research?

Due to time constraints, the author cannot investigate the degree to which
privacy concerns are affecting mobile platform adoption.  Certainly, concepts of
privacy are changing rapidly, but to what degree are people comfortable truly
out sourcing much of their private data is at this point, unknown.  Privacy
concerns may be a critical factor in mobile platform adoption and if that is the
case, designing the perfect mobile platform would have little or no effect on
adoption.

\section{Results}

\subsection{In general, what was found? What happened?}

\subsection{Data provided to justify statements or major trends?}

\subsection{Reliability of the results?}

\subsection{How was reliability demonstrated?}

\end{document}
