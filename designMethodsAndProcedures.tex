\section{Design: Methods and Procedures}

This section outlines the different areas of knowledge that a reader of this
paper will need to be familiar with in order to be able to successfully
interpret the research presented.

The author assumes the reader has knowledge of:
\begin{inparaenum}[(1)]
\item various aspects of hardware performance, especially those associated with
  inter-charge mobile life expectancy;
\item Cloud Services and how they are used;
\item the offerings available currently in today's mobile computing environment.
\end{inparaenum}

The problem that this research project is addressing focuses on the fact that
current adoption of ultra-mobile PCs has primarily been driven by price rather
than mobile performance \citep{ap2009}.  Wider spread adoption by customers
looking to have a device that is optimized for portability and long battery life
rather than just customers that are looking for a cheaper Notebook hasn't been
seen yet.  The author believes that the reason for this is that mobile platform
providers are still designing their machines to primarily behave as a stripped
down desktop.  As such, they do not optimize for use of Cloud Services in such a
way that their device will experience longer battery life and better performance
in that context.  The problem, then, is discovering what factors would have to
coincide to produce an ideal mobile platform for the public consumer.

The research questions being addressed are as follows:
\begin{inparaenum}[(1)]
  \item The author's main investigation is into whether or not today's
    ultra-mobile PCs could be designed differently in order to take advantage of
    Cloud Services in such a way that mobile usage experience would be improved.
    If this is the case and ultra-mobile PCs could be designed differently, what
    would the ideal mobile platform, designed to take advantage of cloud
    services, look like?
  \item A corollary investigation is into whether or not Netbooks and Notebooks
    perform nearly indistinguishably from each other when utilizing cloud
    services despite the observable differences in performance when using local
    applications.
\end{inparaenum}

The procedure that the researcher pursued for this project was split into 2
areas of investigation:
\begin{inparaenum}[(1)]
\item ways that battery life can be maximized through demand management; and
\item Netbook/Notebook performance comparability.
\end{inparaenum}
Battery life is important because it is the author's belief that the main
opportunity for improvement in the ultra-mobile PC space is the extension of
battery life through demand management.  In order to extend battery life,
Netbooks would have to be designed differently to emphasize the use of external
rather than local services.  Netbook/Notebook performance comparability matters
because the author believes that proving that Netbooks and Notebooks perform
indistinguishably from each other when using Cloud Services will reveal that
Netbooks can be stripped down further in order to extend battery life.

The author investigated research that has already been done in the area of
battery life performance.  It is common knowledge that a bright display, WiFi,
and Bluetooth are the biggest killers of battery life.  The author surveyed
research in the Demand Management field to find evidence for those claims.

The primary strategy for investigating Netbook vs. Notebook performance was
testing performance for services offered both locally and in the cloud.  This
was executed by testing spreadsheet operations on Google Docs and doing the same
operations locally.  Time to complete various tasks associated with the
spreadsheet was measured.  Based on this performance research, the author
constructed a theoretical computer that is optimized for the use of these
services in order to gain other performance boosts.

\subsection{Instrumentation}

Equipment needed for this study included
\begin{inparaenum}[(1)]
\item A MacBook;
\item An acer ASPIRE one;
\item Google Chrome;
\item OpenOffice.org;
\item The testing Spreadsheet.\footnote{Spreadsheet published here:
  \url{j.mp/cmscu498GSs}}
\item A Google account with access to Google Docs for spreadsheet analysis.
\item A clock or stopwatch to measure performance\footnote{The researcher used
  iChrono found here: \url{http://widgets.tossen.net/}}.
\end{inparaenum}

The hardware and software requirements to run this test are outlined in detail
in \emph{Table~\ref{table:hardwareSpecs}} on page~\pageref{table:hardwareSpecs}
and \emph{Table~\ref{table:softwareSpecs}} on page~\pageref{table:softwareSpecs}
respectively.

  \begin{table}[tbp]
    \begin{center}
    \begin{tabular}{| r | p{5cm} | p{5cm} |}
      \hline
                            & MacBook                                                    & acer ASPIRE one                                     \\ \hline         
      Model Identifier      & MacBook1,1 1                                               & Acer Aspire one V1.07                               \\ \hline
      Processor Name        & Intel Core Duo                                             & Intel Atom N270                                     \\ \hline
      Processor Speed       & 1.83 GHz                                                   & 1.60 GHz                                            \\ \hline
      Number Of Processors  & 1                                                          & 1                                                   \\ \hline
      Total Number Of Cores & 2                                                          & 1                                                   \\ \hline
      L2 Cache              & 2 MB                                                       & 512 KB                                              \\ \hline
      Main Memory           & 2 GB                                                       & 1 GB                                                \\ \hline
      Operating System      & Mac OS X 10.6.3                                            & Windows XP Home Edition Service Pack 3 (build 2600) \\ \hline
      Network  Adapter      & Marvell Yukon Gigabit Adapter 88E8053 Singleport Copper SA & Atheros AR5007EG Wireless Network Adapter           \\ \hline
      Bus Speed             & 667 MHz                                                    & N/A                                                 \\ \hline
      Boot ROM Version      & MB11.0061.B03                                              & N/A                                                 \\ \hline
      SMC Version (system)  & 1.4f12                                                     & N/A                                                 \\ \hline
      Enclosure Type        & N/A                                                        & Notebook                                            \\ \hline
      Primary Cache         & N/A                                                        & 48 KB                                               \\ \hline
      Hyper-threading       & N/A                                                        & 2 total                                             \\
      \hline
    \end{tabular}
    \caption{Hardware Specifications}
    \label{table:hardwareSpecs}
    \end{center}
  \end{table}


  \begin{table}
    \begin{center}
    \begin{tabular}{| r | p{5cm} | p{5cm} |}
      \hline
                                   & MacBook                      & acer ASPIRE one              \\ \hline
      Google Chrome                & 5.0.342.9                    & 4.1.249.1045 (42898)         \\ \hline
      OpenOffice.org               & 3.2.0 OOO320m12 (Build:9483) & 3.2.0 OOO320m12 (Build:9483) \\
      \hline
    \end{tabular}
    \caption{acer ASPIRE one Software}
    \label{table:softwareSpecs}
    \end{center}
  \end{table}


\subsection{Procedure/Methodology}

\label{sec:proc}
%Expected recipe

The researcher used the following process to test the performance of Netbooks
and Notebooks using Google Docs Spreadsheet and its locally hosted counterpart
OpenOffice.org Calc:

\begin{compactenum}
\item Create a source CSV file with ~3,000 random numbers between -1 and
  1.  \label{item:proc1}
\item Create a new spreadsheet based on the CSV file.  \label{item:proc2}
\item Create a new column and fill with the function =An*10+1 where n is the
  current row.  \label{item:proc3}
\item Create a new column and fill with the function =MEDIAN(B1:Bn) where n is
  the current row.  \label{item:proc4}
\item Create a new column and enter 1-100 in the first hundred
  rows.  \label{item:proc5}
\item Create a new column and fill with the function =Dn/100 where n is the
  current row.  \label{item:proc6}
\item Create a new column and fill with the function =PERCENTILE(C1:C10004,En)
  where n is the current row.  \label{item:proc7}
\item Change an A column cell.  \label{item:proc8}
\end{compactenum}

The researcher used a Groovy script with the argument .25 to generate the random
numbers used to seed the functions of the spreadsheet.  For source see
Listing~\ref{randomSourceGen.groovy} in Appendix~\ref{sourceCode}.

\subsection{Delimitations}

%Suggestions for future research?

Generalization would be difficult because only Intel Core Duo and Atom
processors were tested.  Also, a single broadband Internet connection was used
to conduct the testing so speed differences have not been fully explored.
Different office suites such as Microsoft Office and iWork offer different
performance characteristics and only one has been explored.  However, none of
these considerations should overly affect the ability to generalize across
machines of comparable hardware characteristics.

It is impossible to prove the generalization of this research because there was
not enough time or resources to truly test on varied hardware and over various
connections.  However, the researcher cannot determine any reasons why the
outcomes of the testing would vary too widely across multiple mobile platforms.
Also, performance testing lacked precision as actual process sampling could not
be accomplished.  Instead, observation alone was used in conjunction with the
system monitoring applications of the respective operating systems.

\subsection{Results}

In this section the author will document the results based on the process
outlined in section~\ref{sec:proc} on page \pageref{sec:proc}.  First, the
author will examine the results of following the process in Google Chrome.

On the MacBook in Google Chrome, creation of the spreadsheet from the source
file took an average of 6 seconds based on ten trial uploads on the MacBook in
Google Chrome.  Creation of the second column where the initial number between
-1 and 1 is multiplied by 10 and increased by 1 through a formula fill took an
average of 3 seconds.  The third column where successively larger medians are
taken based on the current row took significantly longer, averaging 25 seconds.
Data can be found in Table~\ref{mbChromePerf} on page~\pageref{mbChromePerf}.
No CPU usage was observed during the creation of these columns except when the
new data had to be displayed.

\begin{table}
  \begin{center}
  \begin{tabular}{| c | l | l | l |}
    \hline
    Run      & Step 1 & Step 2 & Step 3 \\ \hline
    1        & 8      & 2      & 30     \\ \hline
    2        & 7      & 3      & 17     \\ \hline
    3        & 5      & 4      & 24     \\ \hline
    4        & 6      & 3      & 23     \\ \hline
    5        & 6      & 2      & 29     \\ \hline
    6        & 4      & 5      & 28     \\ \hline
    7        & 5      & 2      & 30     \\ \hline
    8        & 6      & 2      & 19     \\ \hline
    9        & 6      & 3      & 20     \\ \hline
    10       & 7      & 3      & 23     \\ \hline
    Average  & 6      & 2.9    & 24.3   \\
    \hline
  \end{tabular}
  \caption{MacBook/Chrome Performance. All step times are in seconds. Steps 1--3 correspond to Steps 2--4 of the Procedure.}
  \label{mbChromePerf}
  \end{center}
\end{table}


Creating the 3 columns corresponding to Steps 5--7 needed to calculate
percentiles was uninteresting.  Each task averaged less than 1 second over 10
runs.

As was expected, performance was extremely comparable on the ASPIRE one when
using Chrome.  While there was a slight performance degradation, time to
complete the various tasks was almost identical to completing them on the
MacBook.  See Table~\ref{aspChrPerf} on page~\pageref{aspChrPerf}.

\begin{table}
  \begin{tabular}{| c | l | l | l |}
    \hline
    Run          & Step 1 & Step 2 & Step 3 \\ \hline
    1            & 8      & 2      & 17     \\ \hline
    2            & 7      & 3      & 30     \\ \hline
    3            & 5      & 2      & 23     \\ \hline
    4            & 6      & 3      & 30     \\ \hline
    5            & 7      & 4      & 20     \\ \hline
    6            & 9      & 5      & 24     \\ \hline
    7            & 3      & 2      & 29     \\ \hline
    8            & 10     & 3      & 22     \\ \hline
    9            & 6      & 3      & 30     \\ \hline
    10           & 7      & 4      & 23     \\ \hline
    Averages     & 6.8    & 3.1    & 24.8   \\
    \hline
  \end{tabular}
  \caption{acer ASPIRE one/Chrome Performance: Step One}
  \label{aspChrPerf}
\end{table}

At the sampling size Google Docs could handle before becoming unusable which was
around the size of the sample spreadsheet (3,000 data cells, 6,300 formula
cells), creating each of the columns of the Spreadsheet was instantaneous
\emph{on both platforms}.  This was very surprising to the researcher although
in retrospect it should not have been.  At that sample size, no local
spreadsheet system has issues.  The researcher obtained a much larger
spreadsheet\footnote{A testing spreadsheet for use with OpenOffice.org bug 89976
  \url{http://j.mp/cmscu498OOoSs}} which allowed a much closer comparison of the
two platforms and as expected the netbook performed far worse than the notebook;
actually an \emph{order of magnitude} worse.  See Table~\ref{ooCp} on
page~\pageref{ooCp}.

\begin{table}
  \begin{center}
  \begin{tabularx}{300pt}{| c | X | X |}
    \hline
                      & \multicolumn{2}{c|}{Time to complete opening in seconds} \\ \hline
    Run               & Notebook & Netbook    \\ \hline
    1                 & 134      & 1800       \\ \hline
    2                 & 129      & 1738       \\ \hline
    3                 & 144      & 1721       \\ \hline
    4                 & 111      & 1983       \\ \hline
    5                 & 119      & 1699       \\ \hline
    6                 & 128      & 1738       \\ \hline
    7                 & 145      & 1715       \\ \hline
    8                 & 133      & 1817       \\ \hline
    9                 & 129      & 1876       \\ \hline
    10                & 125      & 1743       \\ \hline
    Average Times     & 129.7    & 1783       \\
    \hline
  \end{tabularx}
  \caption{OpenOffice.org Cross-Platform Performance Data.}
  \label{ooCp}
  \end{center}
\end{table}


These results are not very arbitrary but they are purely based on observation
which means that future researchers will come up with slightly different
results.  The results are demonstrated for verification by the tables of
performance data included in this text

Based on the above performance data, the following specifications are believed
to be an ideal mobile computer designed to maximize battery life and performance
when accessing the cloud.

The processor could be inexpensive and weak as during the utilization of the
Google Docs Spreadsheet neither machine's CPU usage went above 5\% during
calculations.  By intentionally downgrading the CPU, load on the battery would
be reduced.  This could be reduced even further by the addition of a specialized
GPU that was optimized for the display of web graphics would help the overall
performance as the biggest hit on the processor was during the redisplay of
data.

For storage, a small flash card or SIM chip could be used in place of
traditional hard disks.  This would be used to store the browser and minimal OS
needed to utilize the Cloud Service and would minimize the need to spin up, spin
down, and otherwise manage a physical disk.  No disk access was required during
the running of the performance tests, but utilizing Hibernate between sessions
would be much less battery intensive with the ability to simply write the
contents of memory to a flash disk.

Maintaining a small screen size for the mobile device helps greatly with the
battery expectancy but a higher density screen would be desirable.  This way
higher complexity web applications could be utilized comfortably without
sacrificing battery performance.  Another option would be to implement in
hardware the ability to zoom out and in.  This way low density screens could be
used which would boost battery performance slightly without sacrificing
usability.

A specialized OS and Browser could be developed for this system.  While doing
nothing on both machines the OS alone consumed 1--2\% of available CPU power.
Reducing that to 0 would help with battery life and increase the responsiveness
of the various web applications being utilized.
