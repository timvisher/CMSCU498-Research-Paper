\documentclass[12pt,oneside,letterpaper,titlepage]{article}
\author{Tim Visher\\ CMSCU 498\\ Chestnut Hill College}
\title{Design: Methods and Procedures}

\usepackage[colorlinks=true]{hyperref}
\usepackage{paralist}
\usepackage{listings}
\lstset{basicstyle=\small}

\begin{document}

\maketitle

\tableofcontents

\newpage

\section{Introduction to Research Project}

This section will outline the different areas of knowledge that a reader of this
paper will need to be familiar with on order to be able to successfully
interpret the research presented.

The reader should:

\begin{itemize}

\item be familiar with various aspects of hardware performance, especially those
  associated with inter-charge mobile life expectancy.  Much of the research
  here-in has to do with how computer usage and hardware design affects this
  life expectancy and what can be done to make mobile devices last longer on a
  single charge.

\item be familiar with Cloud Services and how they are used.  The research deals
  heavily with the different demands that Cloud Services make on mobile
  computing hardware compared to traditional desktop equivalents of the same
  services.

\item be familiar with the offerings available currently in today's mobile
  computing environment.  The research will discuss various performance costs
  and benefits in relation to traditional notebooks, netbooks, and smart-phones.

\item a cursory understanding of software testing.  Some of the procedures the
  researcher will be using will center around automating the testing of various
  aspects of computer performance on various mobile platforms as well as in the
  cloud.

\end{itemize}

The problem that this research project is addressing is as follows: Currently,
adoption of ultra-mobile PCs has primarily been driven by price.  The author
believes that the reason wider spread adoption hasn't been seen yet is of a dual
nature:

\begin{enumerate}

\item Mobile platform providers are still designing their machines to primarily
  behave as a stripped down desktop.  As such, they do not optimize for use of
  Cloud Services in such a way that their device will experience longer battery
  life and better performance.

\item Consumers still believe that computers are primarily for executing tasks
  locally.  The author believes that this is no longer true, both from personal
  experience as well as from usage reports.  However, consumer perception is as
  if not more important than actual facts.

\end{enumerate}

The problem, then, is discovering what factors would have to coincide to produce
an ideal mobile platform for the public consumer.

The questions and hypotheses being addressed are as follows:

\begin{enumerate}

  \item The author's main hypothesis is that mass adoption of ultra-mobile PCs
    will not be seen until hardware offerings pass muster.

  Tightly integrated here are 2 questions:

  \begin{enumerate}

    \item What about current mobile platform offerings is unsatisfying to the
      average consumer?

    \item What would the ideal public consumer mobile platform, designed to take
      full advantage of cloud services, look like?

  \end{enumerate}

  \item A corollary hypothesis is that Netbooks and Notebooks perform nearly
    indistinguishably from each other when utilizing cloud services despite the
    observable differences in performance when using local applications.

  \item Another hypothesis that there likely will be no time to examine is that
    average consumer perception of computer use does not adequately reflect
    reality.

  \item A final, tangential question that the author is interested in is to what
    degree privacy concerns are holding back public adoption of cloud services.

\end{enumerate}

The procedure that the researcher wishes to pursue for this project is split
into 2 areas of investigation: (1) hampered mobile platform adoption and; (2)
netbook/notebook performance comparability.

\subsection{Hampered Mobile Platform Adoption}

The author intends to investigate research that has already been done in the
area of battery life performance.  It is somewhat common knowledge that a bright
display, WiFi, and Bluetooth are the biggest killers of battery life.  The
author would like to investigate if there is verifiable truth to that claim.

\subsection{Netbook/Notebook Performance Comparability}

The primary strategy for investigating Netbook vs. Notebook performance will be
testing performance for services offered both locally and in the cloud.

This will attempt to verify that there is indeed little to no difference between
Netbook and Notebook performance when the cloud is what is being utilized.

\section{Instrumentation}

%What do you need to do your project.  Only a couple paragraphs.

The primary participants in this research are 
\begin{inparaenum}[(1)]
\item A MacBook notebook computer;
\item An acer ASPIRE one netbook computer;
\item Google Spreadsheet;
\item Google Chrome; and
\item OpenOffice.org
\end{inparaenum}

Generalization would be difficult because only Intel Core Duo and Atom
Processors are being tested.  Also, a single broadband Internet connection was
used to conduct the testing so speed differences have not been fully explored.
Also, different office suites offer different performance.

Equipment needed to reconduct my study would be
\begin{inparaenum}[(1)]
\item A MacBook;
\item An acer ASPIRE one;
\item Several Browsers;
\item OpenOffice.org;
\item The testing Spreadsheet.
\end{inparaenum}

\begin{enumerate}

\item A MacBook with the following specifications:

  \begin{table}[htbp]
    \caption{MacBook Specifications}
    \begin{tabular}{| r | p{5cm} |}
      \hline
      Model Identifier      & MacBook1,1 1 \\ \hline
      Processor Name        & Intel Core Duo \\ \hline
      Processor Speed       & 1.83 GHz \\ \hline
      Number Of Processors  & 1 \\ \hline
      Total Number Of Cores & 2 \\ \hline
      L2 Cache              & 2 MB \\ \hline
      Memory                & 2 GB \\ \hline
      Bus Speed             & 667 MHz \\ \hline
      Boot ROM Version      & MB11.0061.B03 \\ \hline
      SMC Version (system)  & 1.4f12 \\ \hline
      Mac OS X              & 10.6.3 \\ \hline
      Network  Adapter      & Marvell Yukon Gigabit Adapter 88E8053 Singleport
      Copper SA \\
      \hline
    \end{tabular}
  \end{table}

  with the following software installed at the noted versions:

  \begin{tabular}{| r | p{5cm} |}
    \hline
    Google Chrome   & 5.0.342.9 \\ \hline
    Safari          & 4.0.5 (6531.22.7) \\ \hline
    Mozilla Firefox & Mozilla/5.0 (Macintosh; U; Intel Mac OS X 10.6; en-US;
    rv:1.9.2) Gecko/20100115 Firefox/3.6 \\ \hline
    OpenOffice.org  & 3.2.0 OOO320m12 (Build:9483) \\
    \hline
  \end{tabular}

\item An acer ASPIRE one with the following specifications:

  \begin{table}[htbp]
    \begin{tabular}{| r | p{5cm} |}
      \hline
      Model                        & Acer Aspire one V1.07 \\ \hline
      Enclosure Type               & Notebook \\ \hline
      Operating System             & Windows XP Home Edition Service Pack 3 (build
                                     2600) \\ \hline
      Processor Name               & Intel Atom N270 \\ \hline
      Processor Speed              & 1.60 GHz \\ \hline
      Number of Processors         & 1 \\ \hline
      Total Number of Cores        & 1 \\ \hline
      Primary Cache                & 48 KB \\ \hline
      Secondary Cache              & 512 KB \\ \hline
      Hyper-threading              & 2 total \\ \hline
      Main Memory                  & 1 GB \\ \hline
      Network Adapter              & Atheros AR5007EG Wireless Network Adapter \\
      \hline
    \end{tabular}
  \end{table}

  with the following software installed at the noted versions:

  \begin{tabular}{| r | p{5cm} |}
    \hline
    Google Chrome                & 4.1.249.1045 (42898) \\ \hline
    Safari                       & 4.0.5 (531.22.7) \\ \hline
    Mozilla Firefox              & Mozilla/5.0 (Window; U; Windows NT 5.1;
                                   en-US; rv:1.9.2.3) Gecko/20100401
                                   Firefox/3.6.3 (.NET CLR 3.5.30729) \\ \hline
    OpenOffice.org               & 3.2.0 OOO320m12 (Build:9483) \\
    \hline
  \end{tabular}

% \item Online resources to conduct online research

\item A Google account with access to Google Docs for spreadsheet analysis.

\item The Spreadsheet.\footnote{Spreadsheet published here:
  \url{j.mp/cmscu498GSs}}

\item A clock or stopwatch to measure performance.

\end{enumerate}

It is impossible to generalize this research because there was not enough time
or resources to truly test on varied hardware and over various connection.
Also, performance testing was somewhat arbitrary as actual sampling could not be
accomplished.  Instead, observation alone was used in conjunction with the
system monitoring applications of the respective operating systems.

\section{Procedure/Methodology}

\subsection{Performance Testing}

%Expected recipe

The steps taken by the researcher to test performance are the following:

\begin{enumerate}

\item Create a source CSV file with ~3,000 random numbers between -1 and 1.

\item Create a new spreadsheet based on the CSV file.

\item Create a new column and fill with the function =An*10+1 where n is the
  current row.

\item Create a new column and fill with the function =MEDIAN(B1:Bn) where n is
  the current row.

\item Create a new column and enter 1-100 in the first hundred rows.

\item Create a new column and fill with the function =Dn/100 where n is the
  current row.

\item Create a new column and fill with the function =PERCENTILE(C1:C10004,En)
  where n is the current row.

\item Change an A column cell.

\end{enumerate}

The researcher used the following Groovy script with the argument .25 to
generate the random numbers used to seed the functions of the spreadsheet:

\lstinputlisting[frame=single,breaklines]{../randomSourceGen.groovy}

\subsection{MacBook/Chrome Performance}

Creation of the spreadsheet from the source file took an average of 6 seconds
based on ten trial uploads on the MacBook in Google Chrome:

\begin{tabular}{| c | l |}
  \hline
  Run  & Time \\ \hline
  1    & 8    \\ \hline
  2    & 7    \\ \hline
  3    & 5    \\ \hline
  4    & 6    \\ \hline
  5    & 6    \\ \hline
  6    & 4    \\ \hline
  7    & 5    \\ \hline
  8    & 6    \\ \hline
  9    & 6    \\ \hline
  10   & 7    \\ \hline
       & 6    \\
  \hline
\end{tabular}

Creation of the second column took an average of 3 seconds

\begin{tabular}{| c | l |}
  \hline
  Run  & Time \\ \hline
  1    & 2    \\ \hline
  2    & 3    \\ \hline
  3    & 4    \\ \hline
  4    & 3    \\ \hline
  5    & 2    \\ \hline
  6    & 5    \\ \hline
  7    & 2    \\ \hline
  8    & 2    \\ \hline
  9    & 3    \\ \hline
  10   & 3    \\ \hline
       & 2.9  \\
  \hline
\end{tabular}

The third column took significantly longer, averaging 25 seconds.

\begin{tabular}{| c | l |}
  \hline
  Run  & Time \\ \hline
  1    & 30   \\ \hline
  2    & 17   \\ \hline
  3    & 24   \\ \hline
  4    & 23   \\ \hline
  5    & 29   \\ \hline
  6    & 28   \\ \hline
  7    & 30   \\ \hline
  8    & 19   \\ \hline
  9    & 20   \\ \hline
  10   & 23   \\ \hline
       & 24.3 \\
  \hline
\end{tabular}

The problem seemed to be filling the formula down on the entire column.  Google
Spreadsheet appears to be unable to handle large column selections without
freezing.  The researcher was unable to ascertain why this would be.  No CPU
usage of note was observed and no network traffic to speak of.  On occasion
after half an hour or so the page would refresh and the data would be there, but
more often than not it simply would never show up.

Creating the 3 columns needed to calculate percentiles was uninteresting.  Each
task averaged less than 1 second over 10 runs.

\subsection{ASPIRE one/Chrome Performance}

As was expected, performance was extremely comparable on the ASPIRE one when
using Chrome.

The first column:

\begin{tabular}{| c | l |}
  \hline
  Run  & Time \\ \hline
  1    & 8    \\ \hline
  2    & 7    \\ \hline
  3    & 5    \\ \hline
  4    & 6    \\ \hline
  5    & 7    \\ \hline
  6    & 9    \\ \hline
  7    & 3    \\ \hline
  8    & 10   \\ \hline
  9    & 6    \\ \hline
  10   & 7    \\ \hline
       & 6.8  \\
  \hline
\end{tabular}

The second:

\begin{tabular}{| c | l |}
  \hline
  Run  & Time \\ \hline
  1    & 2    \\ \hline
  2    & 3    \\ \hline
  3    & 2    \\ \hline
  4    & 3    \\ \hline
  5    & 4    \\ \hline
  6    & 5    \\ \hline
  7    & 2    \\ \hline
  8    & 3    \\ \hline
  9    & 3    \\ \hline
  10   & 4    \\ \hline
       & 3.1  \\
  \hline
\end{tabular}

The third:

\begin{tabular}{| c | l |}
  \hline
  Run  & Time \\ \hline
  1    & 17   \\ \hline
  2    & 30   \\ \hline
  3    & 23   \\ \hline
  4    & 30   \\ \hline
  5    & 20   \\ \hline
  6    & 24   \\ \hline
  7    & 29   \\ \hline
  8    & 22   \\ \hline
  9    & 30   \\ \hline
  10   & 23   \\ \hline
       & 24.8 \\
  \hline
\end{tabular}

Creating the 3 columns needed to calculate percentiles remained uninteresting.

As an aside: while no significant difference at all was observed between the
two platforms' ability to compute the calculations demanded by the spreadsheet,
significant differences were observed in the performance associated with
changing the display of the data.  It appears that Google offloads most of the
data into memory and relies on local calculations to redisplay the data in the
window.  With the weaker CPU and Memory set, the Netbook saw performance losses
due to that.

\subsection{OpenOffice.org Performance}

At the sampling size Google Docs could handle before becoming unusable, creating
each of the columns of the Spreadsheet was instantaneous \emph{on both
  platforms}.  This was very surprising to the researcher although in retrospect
it should not have been.  At that sample size, no local spreadsheet system has
issues.  The researcher obtained a much larger spreadsheet\footnote{A testing
  spreadsheet for use with OpenOffice.org bug 89976
  \url{http://j.mp/cmscu498OOoSs}} which allowed a much closer comparison of the
two platforms and as expected the netbook performed far worse than the notebook;
actually an \emph{order of magnitude} worse.

\begin{tabular}{| c | l | l |}
  \hline
  Run  & MacBook & ASPIRE one \\ \hline
  1    & 134     & 1800       \\ \hline
  2    & 129     & 1738       \\ \hline
  3    & 144     & 1721       \\ \hline
  4    & 111     & 1983       \\ \hline
  5    & 119     & 1699       \\ \hline
  6    & 128     & 1738       \\ \hline
  7    & 145     & 1715       \\ \hline
  8    & 133     & 1817       \\ \hline
  9    & 129     & 1876       \\ \hline
  10   & 125     & 1743       \\ \hline
       & 129.7   & 1783       \\
  \hline
\end{tabular}

\section{Delimitations}

%Suggestions for future research?

Initially a much larger spreadsheet was desired but performance limits in Google
Docs forbid this.  Unfortunately, that made comparison at the local level
effectively moot as neither platform showed any difficulty at all in performing
the various calculations while much larger samples (~65,000) would take
significant time to open and just as much time to recalculate if things were
changed.  However, both systems had to be operating on the same data both
locally and in the cloud.

Time and resource constraints don't permit the researcher to test on more
platforms.  It would be interesting to test different hardware configurations.

Time constraints don't permit the author to investigate thoroughly why Google
Spreadsheet performance is what it currently is.  Contacts within Google would
almost certainly be required.

Due to time constraints, the author cannot investigate the degree to which
privacy concerns are affecting mobile platform adoption.  Certainly, concepts of
privacy are changing rapidly, but to what degree are people comfortable truly
out sourcing much of their private data is at this point, unknown.  Privacy
concerns may be a critical factor in mobile platform adoption and if that is the
case, designing the perfect mobile platform would have little or no effect on
adoption.

\section{Results}

In general, the researcher was amazed to find that Google Docs is virtually
unusable in it's present form.  While taking on almost no CPU usage locally at
all, it still took large amounts of time to enter formulas and calculate their
values for a relatively small set (3,000) of data points.  Compared to
OpenOffice.org on either platform, this was extremely slow.  Interestingly
enough, CPU and Bandwidth did not seem to play a factor in Google Docs
performance.  The researcher's only guess is that the way they allocate
computing resources at Google is quite limiting in power.

These results are not very arbitrary but there is some observation in there
which means that future researchers will come up with slightly different
results.  The results are demonstrated for verification by the tables of
performance data included in this text.

\end{document}
