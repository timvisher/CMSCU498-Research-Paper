\section{Discussion}

In general, the researcher found that Google Docs is virtually unusable in its
present form for spreadsheets of any reasonable size.  While consuming almost no
CPU usage locally, it still took large amounts of time to enter formulas and
calculate their values for a relatively small set (3,000) of data points.
Compared to OpenOffice.org on either platform, this was extremely slow.
Interestingly enough, CPU and Bandwidth did not seem to play a factor in Google
Docs performance.  The researcher was unable to ascertain why this is the case.
Further research would be required to discover why.

%%\section{Was the original question answered? This should be a simple statement
%%of the support or lack of support of the original question in the
%%introduction}

Due to the research into the performance of batteries for mobile platforms and
the researcher's performance testing regarding utilizing cloud services with
different mobile platforms, the researcher was able to ascertain a direction in
which developers of mobile platforms could head in order to meet the demands of
todays mobile computing customers.  Specifically, very low CPU power was needed
to do complex calculations via a cloud service on either of the mobile platforms
tested on.  Because the power requirements are so low for the local machine when
performing computation on another machine, a computer could be designed that did
nothing but offload its computational requirements to another machine and
display the data that came back.  While dumb terminals meet some of the demands
of this ideal mobile platform, they are not mobile and they are not designed to
be.  The ideal mobile platform would be a mobile mostly-dumb terminal that could
perform extremely basic tasks locally (manipulating the display of data in
memory, creating small text documents that could exist in memory until synced
with the cloud, etc.) while requiring that almost all disk access and
computation be done in the cloud.  A breakthrough in display technology energy
efficiency needs to be discovered as currently the largest drawer of power in
any mobile platform is the display device.

Specifically, based on the performance data, the following specifications are
believed to be an ideal mobile computer designed to maximize battery life and
performance when accessing the cloud.

Because neither machine's CPU usage went above 5\% during the utilization of the
Google Docs Spreadsheet for calculations, the processor could be more
inexpensive and weak even than the Atom Processor in the Netbook.  By
intentionally downgrading the CPU, load on the battery would be reduced.  This
could be reduced even further by the addition of a specialized GPU that was
optimized for the display of web graphics would help the overall performance as
the biggest hit on the processor was during the redisplay of data.

For storage, a small flash card or SIM chip could be used in place of
traditional hard disks.  This would be used to store the browser and minimal OS
needed to utilize the Cloud Service and would minimize the need to spin up, spin
down, and otherwise manage a physical disk.  No disk access was required during
the running of the performance tests, but utilizing an aggressive Hibernate
schedule between sessions would be much less battery intensive with the ability
to simply write the contents of memory to a flash disk.  Obviously, it isn't
possible to use the computer while it is in Hibernate mode, but it does allow
the battery to last longer between uses.

Maintaining a small screen size for the mobile device helps greatly with the
battery expectancy but a higher density screen would be desirable.  This way
higher complexity web applications could be utilized comfortably without
sacrificing battery performance.  Another option would be to implement in
hardware the ability to zoom out and in.  This way low density screens could be
used which would boost battery performance slightly without sacrificing
usability.

A specialized OS and Browser could be developed for this system.  While doing
nothing on both machines the OS alone consumed 1--2\% of available CPU power.
Reducing idle time CPU usage to 0\% would help with battery life and increase
the responsiveness of the various web applications being utilized.

%%\section{Nonstatistical arguments for the generality of the findings?}

These suggestions for the design of ultra-mobile PCs because they are based on
the performance data collected and there are not large local performance
differences between the computers tested on and any other computers on the
market using similar hardware.  Most machines use LCD displays that have varying
brightness.  Wireless technology has basically standardized on 802.11 which has
very similar power characteristics across the varying implementations
(b/g/n/etc.).  In short, the differences between two Netbooks and Notebooks are
negligible when discussing cloud service performance.  The two are only
obviously different, as proved by the performance tests, when performing work
locally.

%%\section{Nonstatistical arguments for the meaningfulness of the findings?}
%% \section{Answer the question “so the procedure produced these results - so
%% what?” What additional relevance was there?}

The findings are meaningful because they do suggest a better design for mobile
platforms.  They show a primary breakthrough that is needed before mobile
computing will become commonplace (other than battery capacity) is display
technologies that consume less power.  Also, a focusing on the paring down of
local computation in favor of the utilization of cloud services could be used to
extend battery life and thus allow for larger and more consumptive displays.
This would require a sophistication of efforts regarding the ability of
broadband Internet access providers to ensure that performance is what users
have come to expect when compared with desktop and traditional mobile devices as
far as interactivity goes.  These realizations could aid in the development of a
mobile platform designed for today's world.


