\section{Discussion}

In general, the researcher was amazed to find that Google Docs is virtually
unusable in its present form for spreadsheets of any reasonabel size.  While
taking on almost no CPU usage locally at all, it still took large amounts of
time to enter formulas and calculate their values for a relatively small set
(3,000) of data points.  Compared to OpenOffice.org on either platform, this was
extremely slow.  Interestingly enough, CPU and Bandwidth did not seem to play a
factor in Google Docs performance.  The researcher's only guess is that the way
they allocate computing resources at Google isn't optimized for response time.

%%\section{Was the original question answered? This should be a simple statement
%%of the support or lack of support of the original question in the
%%introduction}

The original question was answered.  Due to the research into the performance of
batteries for mobile platforms and the researcher's performance testing
regarding utilizing cloud services with different mobile platforms, the
researcher was able to ascertain a direction in which developers of mobile
platforms should head in order to meet the demands of todays mobile computing
customers.  Specifically, very low CPU power was needed to do complex
calculations via a cloud service on either of the mobile platforms tested on.
Because the power requirements are so low to do computation on another machine,
a computer could be designed that did nothing but offload its computational
requirements to another machine and simply displayed the data that came back.
While dumb terminals meet some of the demands of this ideal mobile platform,
they are not mobile and they are not designed to be.  The ideal mobile platform
would be a mobile mostly-dumb terminal that could perform extremely basic tasks
locally (manipulating the display of data in memory, creating small text
documents that could exist in memory until synced with the cloud, etc.) while
requiring that almost all disk access and computation be done in the cloud.  A
breakthrough would have to be discovered in display technology as currently the
largest drawer of power in any mobile platform is the display device.

%%\section{Nonstatistical arguments for the generality of the findings?}

These findings are general because there are not large differences between the
computers tested on and any other computers on the market today in the space
that was being investigated.  Most machines use LCD displays that have varying
brightness.  Wireless technology has basically standardized on 802.11 which has
very similar power characteristics across the varying implementations
(b/g/n/etc.).  In short, the differences between two netbooks and notebooks are
negligible when discussing cloud service performance.  The two are only
obviously different, as proved by the performance tests, when performing work
locally.

%%\section{Nonstatistical arguments for the meaningfulness of the findings?}
%% \section{Answer the question “so the procedure produced these results - so
%% what?” What additional relevance was there?}

The findings are meaningful because they do lead to a better design for mobile
platforms.  They show that a primary breakthrough (other than battery capacity)
that is needed before mobile computing will become commonplace is display
technologies that consume less power, the display being the most consumputive
component of mobile platforms.  Also, a focussing on the paring down of local
computation in favor of the utilization of cloud services could be used to
extend battery life and thus allow for larger and more consumptive displays.
This would require a sophistication of efforts regarding the volution of
broadband Internet access to ensure that performance is what users have come to
expect when compared with desktop and traditional mobile devices.  These
realizations should aid in the development of a mobile platform designed for
today's world.
