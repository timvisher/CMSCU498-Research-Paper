%% \section{What follow-up could be done on this project?}
%% \section{What other questions could be researched in this area?}

%% From questions and hypotheses

Given time, the author would have tried to consider in detail what about current
mobile offerings is dissatisfying to the average consumer.  Despite enormous
inroads into mobile PC sales made by Netbook manufacturers like ASUS and Acer,
the primary drive for these PCs seems to be cost at this
point\citep{gladstone09}.  There is a dearth of research regarding actual
customer perception of mobile platforms designed for use in conjunction with
cloud services.  Customer surveying could be utilized to answer this question.

Another hypothesis of the researcher is that many people believe they use more
of their computer than they actually do.  At this point the author would have to
measure in some way what demands normal uses makes on a CPU and how that
performance differs between Netbooks and Notebooks.  The author could use the
System Requirements from a variety of popular software, for instance, to get
general benchmarks for what Systems are required to provide hardware-wise at
this point.  The essential argument from many Netbook makers is that if your
primary use of the computer is browsing the Web, writing and receiving e-mail,
and preparing documents and presentations in an office suite, then a Netbook
will provide all the computing performance you need.  The author would've liked
to investigate the validity of this point.

Another question that the author would've liked to answer during this project
was whether considerations surrounding privacy affect peoples' opinion of cloud
services at this point or whether most people are comfortable with a corporation
housing and protecting their private data.  The author imagined that most people
who grew up on the Internet would have a lack of a sense of privacy and so would
not mind giving that data away to be held by someone else that they have no
direct control over.  This would seem to be born out by the heavy use of
services like Facebook to store much of our private information.  The author
also expected the answers to this question to be stratified across generations,
with the older generations being more likely to not trust a service with any of
their data.

%% From design methods and procedures

In regards to performance testing, the following areas of expansion are easily
seen.  Initially a much larger spreadsheet was desired but performance limits in
Google Docs forbid this.  Unfortunately, that made comparison at the local level
effectively moot as neither platform showed any difficulty at all in performing
the various calculations while much larger samples (~65,000) would take
significant time to open and just as much time to recalculate if things were
changed.  However, both systems had to be operating on the same data both
locally and in the cloud.  A different service would have to be discovered that
offered the ability to perform large scale spreadsheet calculations.

Time and resource constraints don't permit the researcher to test on more
platforms.  It would be interesting to test different hardware configurations.

Time constraints don't permit the author to investigate thoroughly why Google
Spreadsheet performance is what it currently is.  Contacts within Google would
almost certainly be required.
