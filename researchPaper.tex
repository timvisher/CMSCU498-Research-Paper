\documentclass[12pt,oneside,letterpaper,titlepage]{report}
\author{Tim Visher\\ CMSCU 498\\ Chestnut Hill College}
\title{Cloud Computing and the Ideal Mobile Platform}

\usepackage[hyperindex=true,colorlinks=true]{hyperref}
\usepackage[all]{hypcap}
\usepackage{natbib}
\usepackage[version=3]{mhchem}
\usepackage[colorlinks=true]{hyperref}
\usepackage{paralist}
\usepackage{listings}
\usepackage{fullpage}

\lstset{basicstyle=\small}

\hypersetup{
    citecolor=blue,
    filecolor=blue,
    linkcolor=blue,
    urlcolor=blue
}

\begin{document}

\maketitle

\tableofcontents

\newpage

\begin{abstract}

Mobile platforms are currently designed to be less powerful but mobile versions
of desktop computers.  At the same time, extraordinary advances have been made
in the area of broadband Internet connectivity.  These advances have lead to the
explosion a software delivery strategy known as Cloud Computing; a resurgence of
the Client/Server computing model so prevalent in the early days of public
computing.  Cloud Services allow a computer to do many operations previously
only thought possible on a powerful local machine by sending the data and
instructions to a much more powerful third party computer and waiting to get the
results back.  The question the researcher is attempting to answer is:
Considering advances in Cloud Computing, what would the ideal mobile platform,
designed for longevity between charges and use of Cloud Services, look like?  In
order to answer this question, two areas of research were investigated: Battery
performance research was surveyed as the researcher believes that battery life is
one of the main culprits behind hampered mobile platform adoption, and
performance testing was done on a traditional Notebook PC and a Netbook PC to
ascertain whether the two devices' use of Cloud Services was identical or not.
The research proved that utilizing Cloud Services requires almost no local
processing power, except in the display of the results while at the same time
confirming that the performance differences between the two mobile platforms was
very noticeable when performing local operations.  The research also indicated
strongly that the main available area of improvement in the design of mobile
platforms is in demand management as research in battery technology especially
has stalled except for a few bright areas like that of utilizing nano technology
to build ultra-capacitive batteries.  The implications of this research lead the
researcher to conclude that the ideal mobile platform would enforce the use of
Cloud Services as a practice of demand management and focus almost all of it's
computational resources on efficiently displaying graphical data and maintaining
fast and reliable network connectivity.

\end{abstract}

\chapter{Introduction}

%% General Introduction

The purpose of this project is to discover the attributes of mobile computing
hardware specifically designed to take advantage of the current resurgence in
computing utilities collectively known as Cloud Computing.  The primary
motivation is that today's Netbook and ultra-mobile PC offerings are still being
designed within a desktop computing frame of mind where most of the work the
computer is going to do is to be taken care of locally.  It is believed that
there is a better design model for these computers in this new environment.  The
author's thesis is that there is a certain performance barrier beyond which
consumers will be willing to re-adopt the thin-client, time-sharing computer
model, that barrier being a combination of the usable life of the device between
charges and optimization for basic tasks performed locally as well as optimum
performance for services accessed via the Internet.  In order to discover this
hardware configuration, the usability of current Cloud Services and hardware
configurations will be investigated.

%% Cloud Computing

Amazon and other high-profile technology companies have already begun to offer a
wide variety of Cloud Services designed to place the brunt of the computational
and storage related burden on computers on the network rather than on the
customers' desktop machines \citep{reiss2008}.  These Cloud Computing services
are giving rise to a new class of applications, termed collectively by Tim
O'Reilly as Web 2.0, that focus on delivering effective software solutions via
the Internet rather than distributing them via shrink-wrapped packages
\citep{bleicher2006}.  Examples of such software included Google's office suite
Google Docs, their mail client GMail, YouTube's video encoding services,
Delicious's bookmarking and tagging services, and Abobe's web-based Photoshop
implementation.  What characterizes each of these services is two fold:
\begin{inparaenum}[(1)]
\item an emphasis on openness and collaboration; and
\item extremely low hardware requirements for the clients accessing them.
\end{inparaenum}
It is the low hardware requirements that are truly interesting for the research
being presented here.  These low requirements allow for a change in hardware
that could produce much longer run times without jeapordizing user experience.

Thus, the move to the Cloud Computing paradigm is happening and is just waiting
for someone to create a hardware platform that is truly optimized for that
environment.

%% Mobile Hardware

In tandem with and because of these developments in Cloud Computing, companies
such as ASUS are beginning to offer what have become known as netbooks or
ultra-mobile PCs.  These computers are stripped down versions of more
traditional notebook computers, in both power and size.  They are designed to
surf the web and do many of the other common things that computer users do
(e-mail, word processing, light gaming) while being very portable and
inexpensive.  Also, devices like the iPhone, Blackberry, and Palm, collectively
known as smart-phones, are revolutionizing the way people surf the web and think
about computing access.  GMail, Google's popular web based e-mail client, has a
version specifically designed to work with such small scale mobile computing
devices (as smart-phones are really much more than phones now) in deference to
the fact that many customers desire to be able to access the services they use
on the Web from anywhere at any time.  At the same time, efforts are being made
to give the experience of a desktop application to all mobile users, an effort
that is very hard to achieve on today's computing platforms (especially
smart-phones).

While these lower power machines are being developed today specifically because
the richness of the Web has put many people into their browser for most of their
day, today's mobile platforms are still being designed to be mobile desktop
computing environments.  This means that they are optimized for local operations
and sacrifice run time to remain performant.  The researcher believes this is
out of step with current trends in Cloud Computing.  Just like in the beginnings
of public computing, Cloud Computing centralizes on the idea of making software
available as a service accessed over cheap or rented hardware.  The problems
faced by the original software providers (public access to the Internet and the
cost of providing and procuring the hardware) are not faced by today's service
providers.  Broadband Internet connectivity is widespread at this point, at
least in the developed and increasingly so in the developing world and hardware
has dropped in price enough that most families are able to afford it.  Despite
these advances, mobile platform providers are still failing to capitalize on the
performance gains possible in the dependence upon Cloud Computing resources
which offload much of the processing power required to do computation to a
foreign machine and thus extend battery life greatly through demand management.
The design of such a device could change the way people of think of mobile
computing.

%% Question

The purpose of this research was to discover how the advent of Cloud Computing
will affect the design of mobile hardware.  Current mobile hardware is designed
with the idea of making a desktop computer mobile.  The emphasis is on local
computational power with applications and services that are hosted on your
machine.  The advent of Cloud Computing could allow for a redesign of mobile
platforms that focused on providing long battery life and high quality
connection to and use of Cloud Services rather than one that focuses on local
services that can also access the cloud as an aside.

The goal of the design would be to
\begin{inparaenum}[(1)]
\item Maximize battery life;
\item Maximize performance, perhaps through the design of a special processor
  that focuses on doing computation specific to utilizing web services
  efficiently; and
\item Minimize local computation to
  \begin{inparaenum}[(a)]
  \item displaying the UI;
  \item translating input; and
  \item maintaining network connectivity.
  \end{inparaenum}
\end{inparaenum}

%% Relevancy

The reason this research question is relevant is that Cloud Services are finally
beginning to gain real traction in the public's eye.  Many people are happy to
get their e-mail primarily through a web interface like GMail or Hotmail,
signaling a real change in public perceptions of performance, customizability,
and privacy.  Services like Google's Office Suite Google Docs, while still in
need of performance improvements, do perfectly serve for smaller spreadsheets,
documents, and presentations.  And they can be used on any platform with access
to a browser with exactly the same performance characteristics as on the most
performant machines today.  This reality allows for the design of a truly cloud
based mobile platform and reduces the barrier for entry for hardware owners to
extremely minimal machines with access to the Internet.

\chapter{Literature Review}

In light of the developments in Cloud Computing, this literature review will be
focused on a variety of topics all germane to the development of ultra-mobile
computers specifically designed for long, cell-phone like inter-charge life
expectancies and the use of Cloud Services.  One historical development that
needs to be covered is Time Sharing.  Time Sharing was a computing model born in
the days when computer hardware was very expensive.  The idea was that a
computer's computational time could be split between many users accessing the
machine over dumb terminals that did nothing but display output and send input.
The idea never really took off in the consumer sector, but it appears to be
making its comeback now.  In conjunction with this is the concept of a
thin-client, which despite the relative lack of a success of the time sharing
computing model has been very successful, especially in the medical space.
Thin-Client hardware is basically a dumb terminal optimized to display output
from a centralized server which may run any kind of operating system.  A great
deal of research has been done on the best way to implement thin-client
architectures, but the field is alive and well and very relevant to the
development of ultra-mobile PCs.

Rechargeable batteries play a significant role in almost every mobile electronic
device on the market today.  Often, battery life is the key component in the
consumers decision making process.  The iPhone was criticized harshly until a
firmware update enabled customers to get a day's worth of use out of it between
charges.  Google's G1 sank under similar criticisms.  Today's ultra-mobiles
don't offer battery life expectancy much greater than today's notebooks.  The
success of smart-phone technology over the ultra-mobile at the moment mainly
hinges on this distinction.

Cloud Computing, Web 2.0, and the developments concerning broadband Internet
access and Internet2 are also topics this literature review will be covering.
Cloud Computing is, as stated, a rehashing of the old time sharing model only
with much higher bandwidth and much broader access.  Web 2.0 is a loose
consortium of ideals that many of the companies providing their software as a
service adhere to \citep{oreilly2007}.  Broadband Internet is self explanatory.
These three technologies together provide the basis for why an ultra-mobile,
thin-client PC could be possible now when it could not have been 10 or 20 years
ago.  These technologies and the literature associated with them will be covered
in the following sections.

  \include{literatureReview}

\chapter{Research Project}

For the research portion of this project, two areas of investigation were explored:
\begin{inparaenum}[(1)]
\item Cloud Service performance using representatives of Netbook and Notebook
  hardware; and
\item Surveying battery performance research surrounding demand management as
  a way to extend battery life.
\end{inparaenum}
The research compiled can be found in section~\ref{sec:batteryLiterature} on
page~\pageref{sec:batteryLiterature}.  The performance research, on the other
hand, is original research and is outlined in detail in the following sections.

The performance testing consisted of utilizing Google Docs' Spreadsheet service
and OpenOffice.org to manipulate a test spreadsheet via a Notebook (MacBook) and
Netbook (asus ASPIRE one) computer.  Performance was observed via time to
complete operation measurements and CPU/Memory usages.

  My primary hypothesis is that mass adoption of ultra-mobile PCs will not happen
till customer hardware requirements are figured out.  Currently, the best
ultra-mobile PC battery life is 14 hours\footnote{Retrieved January 31, 2010
from \url{http://www.netbookreviews.net/asus/eee-pc-1005pe-review/}}                 %TODO
from the Eee PC 1005PE made by Netbook specialists ASUS.  That is not long           %Enter
enough to allow it to be on all day without charging.  Without hardware              %into
providing both the performance necessary to run today's cloud applications and       %bibliography
the inter-charge life expectancy to rival todays smart phones, Netbooks will
remain a niche product unless their price alone can be made compelling enough
that people accept the performance hit and begin to use them as their sole PC.

In close association with this hypothesis, the author intends to address the
following question: What would an ideal mobile offering for the average consumer
look like?  If it is indeed true that consumers still prefer traditional PCs
over Netbooks and Smart-Phones, what would get people to switch to more of a
mobile platform?  Would it be longer battery life?  Higher density screens?
Better software performance?  Would it matter if that performance was offered
via that cloud or would it have to be local?

These questions are especially pertinent to the intended outcome of the research
project.  The author intends to create a paper prototype of a mobile platform
that is ideal to meet average consumer needs.

A second hypothesis is that Netbooks and traditional Notebook PCs are comparable
in the use of cloud services while they are unfavorably comparable in their
performance for traditional desktop usage.  In other words, when using a cloud
service, Notebooks and Netbooks will be just as performant, while doing the same
operations locally will produce a marked difference in performance.  This could
be proven by testing various cloud services and doing the same operations
locally.  A video could be encoded an a Notebook, a Netbook, and through both on
a service such as Vimeo or YouTube.  Time to completion could be
tested. Similarly, a large spreadsheet could be constructed on the 2 devices and
then on Google Docs, and performance could be compared.


  \documentclass[12pt,oneside,letterpaper]{article}
\author{Tim Visher\\ CMSCU 498\\ Chestnut Hill College}
\title{Design: Methods and Procedures}

\usepackage[colorlinks=true]{hyperref}

\begin{document}

\maketitle

\tableofcontents

\section{Introduction to Research Project}

\subsection{Background necessary to understand the research}

\subsection{Reiteration of problem statement, hypothesis or questions being
  addressed}

\subsection{Overview of Procedure}

%Case for why the procedure is appropriate and the best procedure given the
%situtation.

\section{Subjects/Participants}

\subsection{Information necessary to replicate the study with respect to
  subjects}

\subsection{Information necessary to realize inability to generalize because of
  some special property of the subjects}

\section{Instrumentation}

\subsection{Information necessary to exactly replicate the study with respect to
  apparutus or setting}

\subsection{Information necessary to realize inability to generalize beacuse of
  some special property of the apparatus or the setting}

\section{Procedure/Methodology}

\subsection{Specific procedure used to answer the problem}

\subsection{Information necessary to exactly replicate the study with respect to
  the procedure}

\subsection{Procedures used to deal with unavoidable problems}

\subsection{Information necessary to realize inability to generalize because of
  some special property of the procedure}

\subsection{Was a baseline obtained or were the groups equal at the start?}

\subsection{How were the independent variables measured?}

\subsection{How were the dependent variables measured}

\section{Delimitations}

\section{Results}

\subsection{In general, what was found? What happened?}

\subsection{Data provided to justify statements or major trends?}

\subsection{Reliability of the results?}

\subsection{How was reliability demonstrated?}

\end{document}


  \section{Discussion}

In general, the researcher was amazed to find that Google Docs is virtually
unusable in its present form for spreadsheets of any reasonabel size.  While
taking on almost no CPU usage locally at all, it still took large amounts of
time to enter formulas and calculate their values for a relatively small set
(3,000) of data points.  Compared to OpenOffice.org on either platform, this was
extremely slow.  Interestingly enough, CPU and Bandwidth did not seem to play a
factor in Google Docs performance.  The researcher's only guess is that the way
they allocate computing resources at Google isn't optimized for response time.

%%\section{Was the original question answered? This should be a simple statement
%%of the support or lack of support of the original question in the
%%introduction}

The original question was answered.  Due to the research into the performance of
batteries for mobile platforms and the researcher's performance testing
regarding utilizing cloud services with different mobile platforms, the
researcher was able to ascertain a direction in which developers of mobile
platforms should head in order to meet the demands of todays mobile computing
customers.  Specifically, very low CPU power was needed to do complex
calculations via a cloud service on either of the mobile platforms tested on.
Because the power requirements are so low to do computation on another machine,
a computer could be designed that did nothing but offload its computational
requirements to another machine and simply displayed the data that came back.
While dumb terminals meet some of the demands of this ideal mobile platform,
they are not mobile and they are not designed to be.  The ideal mobile platform
would be a mobile mostly-dumb terminal that could perform extremely basic tasks
locally (manipulating the display of data in memory, creating small text
documents that could exist in memory until synced with the cloud, etc.) while
requiring that almost all disk access and computation be done in the cloud.  A
breakthrough would have to be discovered in display technology as currently the
largest drawer of power in any mobile platform is the display device.

%%\section{Nonstatistical arguments for the generality of the findings?}

These findings are general because there are not large differences between the
computers tested on and any other computers on the market today in the space
that was being investigated.  Most machines use LCD displays that have varying
brightness.  Wireless technology has basically standardized on 802.11 which has
very similar power characteristics across the varying implementations
(b/g/n/etc.).  In short, the differences between two netbooks and notebooks are
negligible when discussing cloud service performance.  The two are only
obviously different, as proved by the performance tests, when performing work
locally.

%%\section{Nonstatistical arguments for the meaningfulness of the findings?}
%% \section{Answer the question “so the procedure produced these results - so
%% what?” What additional relevance was there?}

The findings are meaningful because they do lead to a better design for mobile
platforms.  They show that a primary breakthrough (other than battery capacity)
that is needed before mobile computing will become commonplace is display
technologies that consume less power, the display being the most consumputive
component of mobile platforms.  Also, a focussing on the paring down of local
computation in favor of the utilization of cloud services could be used to
extend battery life and thus allow for larger and more consumptive displays.
This would require a sophistication of efforts regarding the volution of
broadband Internet access to ensure that performance is what users have come to
expect when compared with desktop and traditional mobile devices.  These
realizations should aid in the development of a mobile platform designed for
today's world.


  %% \section{What follow-up could be done on this project?}
%% \section{What other questions could be researched in this area?}

%% From questions and hypotheses

Given time, the author would have tried to consider in detail what about current
mobile offerings is dissatisfying to the average consumer.  Despite enormous
inroads into mobile PC sales made by Netbook manufacturers like ASUS and Acer,
the primary drive for these PCs seems to be cost at this
point\citep{gladstone09}.  There is a dearth of research regarding actual
customer perception of mobile platforms designed for use in conjunction with
cloud services.  Customer surveying could be utilized to answer this question.

Another hypothesis of the researcher is that many people believe they use more
of their computer than they actually do.  At this point the author would have to
measure in some way what demands normal uses makes on a CPU and how that
performance differs between Netbooks and Notebooks.  The author could use the
System Requirements from a variety of popular software, for instance, to get
general benchmarks for what Systems are required to provide hardware-wise at
this point.  The essential argument from many Netbook makers is that if your
primary use of the computer is browsing the Web, writing and receiving e-mail,
and preparing documents and presentations in an office suite, then a Netbook
will provide all the computing performance you need.  The author would've liked
to investigate the validity of this point.

Another question that the author would've liked to answer during this project
was whether considerations surrounding privacy affect peoples' opinion of cloud
services at this point or whether most people are comfortable with a corporation
housing and protecting their private data.  The author imagined that most people
who grew up on the Internet would have a lack of a sense of privacy and so would
not mind giving that data away to be held by someone else that they have no
direct control over.  This would seem to be born out by the heavy use of
services like Facebook to store much of our private information.  The author
also expected the answers to this question to be stratified across generations,
with the older generations being more likely to not trust a service with any of
their data.

%% From design methods and procedures

In regards to performance testing, the following areas of expansion are easily
seen.  Initially a much larger spreadsheet was desired but performance limits in
Google Docs forbid this.  Unfortunately, that made comparison at the local level
effectively moot as neither platform showed any difficulty at all in performing
the various calculations while much larger samples (~65,000) would take
significant time to open and just as much time to recalculate if things were
changed.  However, both systems had to be operating on the same data both
locally and in the cloud.  A different service would have to be discovered that
offered the ability to perform large scale spreadsheet calculations.

Time and resource constraints don't permit the researcher to test on more
platforms.  It would be interesting to test different hardware configurations.

Time constraints don't permit the author to investigate thoroughly why Google
Spreadsheet performance is what it currently is.  Contacts within Google would
almost certainly be required.


\appendix

\chapter{Source Code}

This Appendix contains source code used in the process of completing the
research for this project.

\newpage

\label{sourceCode}
%\section{Source code from computer programs or other project}

\lstinputlisting[numbers=left,title=\lstname,frame=single,breaklines,label=randomSourceGen.groovy,caption=randomSourceGen.groovy]{randomSourceGen.groovy}

\newpage
\bibliographystyle{newapa}
\bibliography{bibliography}

\end{document}
