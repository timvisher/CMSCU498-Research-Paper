\documentclass[12pt,oneside,letterpaper,titlepage]{report}
\author{Tim Visher\\ CMSCU 498\\ Chestnut Hill College}
\title{Cloud Computing and the Ideal Mobile Platform}

\usepackage[hyperindex=true,colorlinks=true]{hyperref}
\usepackage[all]{hypcap}
\usepackage{natbib}
\usepackage[version=3]{mhchem}
\usepackage[colorlinks=true]{hyperref}
\usepackage{paralist}
\usepackage{listings}
\usepackage{fullpage}

\lstset{basicstyle=\small}

\hypersetup{
    citecolor=blue,
    filecolor=blue,
    linkcolor=blue,
    urlcolor=blue
}

\begin{document}

\maketitle

\tableofcontents

\newpage

\begin{abstract}

%  \section{Context of research question}
%
%  \section{Explicit statement of question}
%
%  \section{What and how many subjects used}
%
%  \section{How question was answered and Research design}
%
%  \section{Results of research}
%
%  \section{Implication of results}

Lorem ipsum dolor sit amet, consectetur adipisicing elit, sed do eiusmod tempor
incididunt ut labore et dolore magna aliqua. Ut enim ad minim veniam, quis
nostrud exercitation ullamco laboris nisi ut aliquip ex ea commodo
consequat. Duis aute irure dolor in reprehenderit in voluptate velit esse cillum
dolore eu fugiat nulla pariatur. Excepteur sint occaecat cupidatat non proident,
sunt in culpa qui officia deserunt mollit anim id est laborum.

\end{abstract}

\chapter{Introduction}

\newpage

Today's mobile platforms are designed to be mobile desktop computing
environments.  This means that they are optimized for local operations and not
for long run times.  The researcher believes this is out of step with the
current trends in Cloud Computing.  Just like in the beginnings of public
computing, Cloud Computing centralizes on the idea of making software available
as a service accessed over cheap or rented hardware.  The problems faced by the
original software providers (public access to the Internet and the cost of
providing and procuring the hardware) are not faced by today's service
providers, though.  Broadband Internet connectivity is widespread at this point,
at least in the developed and increasingly so in the developing world and
hardware has dropped in price enough that most families are able to afford it,
even in it's hybridized state.  The problem remains that truly widespread mobile
platform adoption is yet to happen.  The researcher believes that this is due to
the failing of mobile platform providers to captilize on the performance gains
possible in the dependence upon cloud computing resources which offload much of
the processing power required to do computation to a foreign machine and thus
extend battery life greatly through demand management.

The purpose of this research was to discover how the advent of cloud computing
will affect the design of mobile hardware.  Current mobile hardware is designed
with the idea of making a desktop computer mobile.  The emphasis is on local
computational power with applications and services that are hosted on your
machine.  The advent of Cloud Computing could allow for a redesign of mobile
platforms that focused on providing long battery life and high quality
connection to and use of cloud services.

The goal of the design would be to 
\begin{inparaenum}[(1)]
\item Maximize battery life;
\item Maximize performance, perhaps through the design of a special processor
  that focusses on doing computation specific to utilizing web services
  efficiently; and
\item Minimize local computation to 
  \begin{inparaenum}[(a)]
  \item displaying the UI;
  \item translating input; and
  \item maintaining network connectivity.
  \end{inparaenum}
\end{inparaenum}

% Case for why the question is important or interesting and why the reader should want an answer?

The reason my research question is relevant is that cloud services are finally
beginning to gain real traction in the public's eye.  Many people are happy to
get their e-mail primarily through a web interface like GMail or Hotmail,
signaling a real change in public perceptions of performance, customizabliity,
and privacy.  Services like Google's Office Suite Google Docs, while still in
need of performance improvements, still perfectly serve for smaller
spreadsheets, documents, and presentations.  And they can be used on any
platform with access to a browser with exactly the same performance
characteristics as on the most performant machines today.  This reality allows
for the design of a truly cloud based mobile platform.

\chapter{Literature Review}

\newpage

  \include{literatureReview}

\chapter{Questions and Hypotheses}

\newpage

  My primary hypothesis is that mass adoption of ultra-mobile PCs will not happen
till customer hardware requirements are figured out.  Currently, the best
ultra-mobile PC battery life is 14 hours\footnote{Retrieved January 31, 2010
from \url{http://www.netbookreviews.net/asus/eee-pc-1005pe-review/}}                 %TODO
from the Eee PC 1005PE made by Netbook specialists ASUS.  That is not long           %Enter
enough to allow it to be on all day without charging.  Without hardware              %into
providing both the performance necessary to run today's cloud applications and       %bibliography
the inter-charge life expectancy to rival todays smart phones, Netbooks will
remain a niche product unless their price alone can be made compelling enough
that people accept the performance hit and begin to use them as their sole PC.

In close association with this hypothesis, the author intends to address the
following question: What would an ideal mobile offering for the average consumer
look like?  If it is indeed true that consumers still prefer traditional PCs
over Netbooks and Smart-Phones, what would get people to switch to more of a
mobile platform?  Would it be longer battery life?  Higher density screens?
Better software performance?  Would it matter if that performance was offered
via that cloud or would it have to be local?

These questions are especially pertinent to the intended outcome of the research
project.  The author intends to create a paper prototype of a mobile platform
that is ideal to meet average consumer needs.

A second hypothesis is that Netbooks and traditional Notebook PCs are comparable
in the use of cloud services while they are unfavorably comparable in their
performance for traditional desktop usage.  In other words, when using a cloud
service, Notebooks and Netbooks will be just as performant, while doing the same
operations locally will produce a marked difference in performance.  This could
be proven by testing various cloud services and doing the same operations
locally.  A video could be encoded an a Notebook, a Netbook, and through both on
a service such as Vimeo or YouTube.  Time to completion could be
tested. Similarly, a large spreadsheet could be constructed on the 2 devices and
then on Google Docs, and performance could be compared.


\chapter{Design: Methods and Procedures}

\newpage

  \documentclass[12pt,oneside,letterpaper]{article}
\author{Tim Visher\\ CMSCU 498\\ Chestnut Hill College}
\title{Design: Methods and Procedures}

\usepackage[colorlinks=true]{hyperref}

\begin{document}

\maketitle

\tableofcontents

\section{Introduction to Research Project}

\subsection{Background necessary to understand the research}

\subsection{Reiteration of problem statement, hypothesis or questions being
  addressed}

\subsection{Overview of Procedure}

%Case for why the procedure is appropriate and the best procedure given the
%situtation.

\section{Subjects/Participants}

\subsection{Information necessary to replicate the study with respect to
  subjects}

\subsection{Information necessary to realize inability to generalize because of
  some special property of the subjects}

\section{Instrumentation}

\subsection{Information necessary to exactly replicate the study with respect to
  apparutus or setting}

\subsection{Information necessary to realize inability to generalize beacuse of
  some special property of the apparatus or the setting}

\section{Procedure/Methodology}

\subsection{Specific procedure used to answer the problem}

\subsection{Information necessary to exactly replicate the study with respect to
  the procedure}

\subsection{Procedures used to deal with unavoidable problems}

\subsection{Information necessary to realize inability to generalize because of
  some special property of the procedure}

\subsection{Was a baseline obtained or were the groups equal at the start?}

\subsection{How were the independent variables measured?}

\subsection{How were the dependent variables measured}

\section{Delimitations}

\section{Results}

\subsection{In general, what was found? What happened?}

\subsection{Data provided to justify statements or major trends?}

\subsection{Reliability of the results?}

\subsection{How was reliability demonstrated?}

\end{document}


\chapter{Discussion}

\newpage

  \section{Discussion}

In general, the researcher was amazed to find that Google Docs is virtually
unusable in its present form for spreadsheets of any reasonabel size.  While
taking on almost no CPU usage locally at all, it still took large amounts of
time to enter formulas and calculate their values for a relatively small set
(3,000) of data points.  Compared to OpenOffice.org on either platform, this was
extremely slow.  Interestingly enough, CPU and Bandwidth did not seem to play a
factor in Google Docs performance.  The researcher's only guess is that the way
they allocate computing resources at Google isn't optimized for response time.

%%\section{Was the original question answered? This should be a simple statement
%%of the support or lack of support of the original question in the
%%introduction}

The original question was answered.  Due to the research into the performance of
batteries for mobile platforms and the researcher's performance testing
regarding utilizing cloud services with different mobile platforms, the
researcher was able to ascertain a direction in which developers of mobile
platforms should head in order to meet the demands of todays mobile computing
customers.  Specifically, very low CPU power was needed to do complex
calculations via a cloud service on either of the mobile platforms tested on.
Because the power requirements are so low to do computation on another machine,
a computer could be designed that did nothing but offload its computational
requirements to another machine and simply displayed the data that came back.
While dumb terminals meet some of the demands of this ideal mobile platform,
they are not mobile and they are not designed to be.  The ideal mobile platform
would be a mobile mostly-dumb terminal that could perform extremely basic tasks
locally (manipulating the display of data in memory, creating small text
documents that could exist in memory until synced with the cloud, etc.) while
requiring that almost all disk access and computation be done in the cloud.  A
breakthrough would have to be discovered in display technology as currently the
largest drawer of power in any mobile platform is the display device.

%%\section{Nonstatistical arguments for the generality of the findings?}

These findings are general because there are not large differences between the
computers tested on and any other computers on the market today in the space
that was being investigated.  Most machines use LCD displays that have varying
brightness.  Wireless technology has basically standardized on 802.11 which has
very similar power characteristics across the varying implementations
(b/g/n/etc.).  In short, the differences between two netbooks and notebooks are
negligible when discussing cloud service performance.  The two are only
obviously different, as proved by the performance tests, when performing work
locally.

%%\section{Nonstatistical arguments for the meaningfulness of the findings?}
%% \section{Answer the question “so the procedure produced these results - so
%% what?” What additional relevance was there?}

The findings are meaningful because they do lead to a better design for mobile
platforms.  They show that a primary breakthrough (other than battery capacity)
that is needed before mobile computing will become commonplace is display
technologies that consume less power, the display being the most consumputive
component of mobile platforms.  Also, a focussing on the paring down of local
computation in favor of the utilization of cloud services could be used to
extend battery life and thus allow for larger and more consumptive displays.
This would require a sophistication of efforts regarding the volution of
broadband Internet access to ensure that performance is what users have come to
expect when compared with desktop and traditional mobile devices.  These
realizations should aid in the development of a mobile platform designed for
today's world.


\chapter{Possible Future Research}

\newpage

  %% \section{What follow-up could be done on this project?}
%% \section{What other questions could be researched in this area?}

%% From questions and hypotheses

Given time, the author would have tried to consider in detail what about current
mobile offerings is dissatisfying to the average consumer.  Despite enormous
inroads into mobile PC sales made by Netbook manufacturers like ASUS and Acer,
the primary drive for these PCs seems to be cost at this
point\citep{gladstone09}.  There is a dearth of research regarding actual
customer perception of mobile platforms designed for use in conjunction with
cloud services.  Customer surveying could be utilized to answer this question.

Another hypothesis of the researcher is that many people believe they use more
of their computer than they actually do.  At this point the author would have to
measure in some way what demands normal uses makes on a CPU and how that
performance differs between Netbooks and Notebooks.  The author could use the
System Requirements from a variety of popular software, for instance, to get
general benchmarks for what Systems are required to provide hardware-wise at
this point.  The essential argument from many Netbook makers is that if your
primary use of the computer is browsing the Web, writing and receiving e-mail,
and preparing documents and presentations in an office suite, then a Netbook
will provide all the computing performance you need.  The author would've liked
to investigate the validity of this point.

Another question that the author would've liked to answer during this project
was whether considerations surrounding privacy affect peoples' opinion of cloud
services at this point or whether most people are comfortable with a corporation
housing and protecting their private data.  The author imagined that most people
who grew up on the Internet would have a lack of a sense of privacy and so would
not mind giving that data away to be held by someone else that they have no
direct control over.  This would seem to be born out by the heavy use of
services like Facebook to store much of our private information.  The author
also expected the answers to this question to be stratified across generations,
with the older generations being more likely to not trust a service with any of
their data.

%% From design methods and procedures

In regards to performance testing, the following areas of expansion are easily
seen.  Initially a much larger spreadsheet was desired but performance limits in
Google Docs forbid this.  Unfortunately, that made comparison at the local level
effectively moot as neither platform showed any difficulty at all in performing
the various calculations while much larger samples (~65,000) would take
significant time to open and just as much time to recalculate if things were
changed.  However, both systems had to be operating on the same data both
locally and in the cloud.  A different service would have to be discovered that
offered the ability to perform large scale spreadsheet calculations.

Time and resource constraints don't permit the researcher to test on more
platforms.  It would be interesting to test different hardware configurations.

Time constraints don't permit the author to investigate thoroughly why Google
Spreadsheet performance is what it currently is.  Contacts within Google would
almost certainly be required.


\appendix

\chapter{Source Code}

\newpage

\label{sourceCode}
%\section{Source code from computer programs or other project}

\lstinputlisting[numbers=left,title=\lstname,frame=single,breaklines,label=randomSourceGen.groovy,caption=randomSourceGen.groovy]{randomSourceGen.groovy}

\chapter{Notes}

%\section{Notes from observations, e-mails, telephone conversations}

\newpage
\bibliographystyle{newapa}
\bibliography{bibliography}

\end{document}
