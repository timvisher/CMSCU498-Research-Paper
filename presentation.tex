% $Header: /cvsroot/latex-beamer/latex-beamer/solutions/generic-talks/generic-ornate-15min-45min.en.tex,v 1.5 2007/01/28 20:48:23 tantau Exp $
\documentclass{beamer}
\mode<presentation>
{
 \usetheme{Warsaw}
 \usecolortheme{beaver}
 \setbeamercovered{transparent}
}
\usepackage[english]{babel}
\usepackage[latin1]{inputenc}
\usepackage{times}
\usepackage[T1]{fontenc}

\title{Cloud Computing and the Ideal Mobile Platform}
\subtitle{How the advent of Cloud Computing might affect computers designed for
  mobile use}
\author{T.~Visher}
\institute[Chestnut Hill College]{Department of Computer Science\\
  Chestnut Hill College}
\date[Senior Seminar]{May 3rd, 2010 / Senior Seminar Presentation}
\subject{Cloud Computing and the Ideal Mobile Platform}

\pgfdeclareimage[height=0.75cm]{chc-logo}{redgrif}
\logo{\pgfuseimage{chc-logo}}

% Delete this, if you do not want the table of contents to pop up at
% the beginning of each subsection:
\AtBeginSubsection[]{
  \begin{frame}<beamer>{Outline}
    \tableofcontents[currentsection,currentsubsection]
  \end{frame}
}

% If you wish to uncover everything in a step-wise fashion, uncomment
% the following command: 

%\beamerdefaultoverlayspecification{<+->}

\begin{document}

\begin{frame}
\titlepage
\end{frame}

\begin{frame}{Outline}
\tableofcontents
% You might wish to add the option [pausesections]
\end{frame}

% Since this a solution template for a generic talk, very little can
% be said about how it should be structured. However, the talk length
% of between 15min and 45min and the theme suggest that you stick to
% the following rules:

% - Exactly two or three sections (other than the summary).
% - At *most* three subsections per section.
% - Talk about 30s to 2min per frame. So there should be between about
%   15 and 30 frames, all told.

\section{Introduction}

\subsection{Goals}

\begin{frame}{Make Titles Informative. Use Uppercase Letters.}{Subtitles are optional.}

\begin{itemize}
  \item Use \texttt{itemize} a lot.
  \item Use very short sentences or short phrases.
\end{itemize}

\end{frame}

\begin{frame}{Make Titles Informative.}
You can create overlays\dots
\begin{itemize}
  \item using the \texttt{pause} command:
    \begin{itemize}
      \item First item. \pause
      \item Second item.
    \end{itemize}
  \item using overlay specifications:
    \begin{itemize}
      \item<3-> First item.
      \item<4-> Second item.
    \end{itemize}
  \item using the general \texttt{uncover} command:
    \begin{itemize}
      \uncover<5->{\item First item.}
      \uncover<6->{\item Second item.}
    \end{itemize}
\end{itemize}
\end{frame}

\subsection{Research Questions}

\begin{frame}{Make Titles Informative.}
\end{frame}

\begin{frame}{Make Titles Informative.}
\end{frame}

\section[Literature Review]{The Literature: Thin Clients, Cloud Services, and Mobile Computers}

\subsection{Thin Client Computing}

\begin{frame}{Make Titles Informative.}
\end{frame}

\subsection{Cloud Computing}

\begin{frame}{Make Titles Informative.}
\end{frame}

\subsection{Netbooks and Ultra-Mobile PCs}

\begin{frame}{Make Titles Informative.}
\end{frame}

\section[Research Project]{The Project: Demand Management and Google Docs Spreadsheet Torture Test}

\subsection{Demand Management}

\begin{frame}{Make Titles Informative.}
\end{frame}

\subsection[GDocs Spreadsheet Testing]{Google Docs Spreadsheet Torture Test}

\begin{frame}{Make Titles Informative.}
\end{frame}

\section*{Summary}

\begin{frame}{Summary}

\begin{itemize}
  \item The \alert{first main message} of your talk in one or two lines.
  \item The \alert{second main message} of your talk in one or two lines.
  \item Perhaps a \alert{third message}, but not more than that.
\end{itemize}

\end{frame}

\begin{frame}{Future Research}
% The following outlook is optional.
\begin{itemize}
  \item Outlook
  \begin{itemize}
    \item Something you haven't solved.
    \item Something else you haven't solved.
  \end{itemize}
  \end{itemize}
\end{frame}

\end{document}
