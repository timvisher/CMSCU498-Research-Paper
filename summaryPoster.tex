\documentclass[letterpaper]{article}
\author{Tim Visher\\ CMSCU 498\\ Chestnut Hill College}
\title{Cloud Computing and the Ideal Mobile Platform}

\usepackage{fullpage}

\begin{document}

\Huge{
\begin{center}
  \emph{Hypotheses}
\end{center}

\begin{enumerate}

\item Mass adoption of ultra-mobile PCs will not happen because of current
  failings in battery life and broadband access.

\item Netbooks and traditional Notebooks are favorably comparable in the use of
  cloud services and unfavorably so in traditional local usage.

\end{enumerate}

\newpage

\begin{center}

\emph{Goal}

\end{center}

Design an ideal mobile platform that is optimized for the use of cloud services
to reduce load on the local CPU and thus increase battery life expectancy.

\newpage

\begin{center}

  \emph{Research Questions}

\end{center}

\begin{enumerate}

\item What about current ultra-mobile PC offerings is dissatisfying to the
  average consumer?

\item What would the ideal mobile platform look like for the average consumer?

\item Is it true that Netbooks perform as well as Notebooks do when utilizing cloud services?

\item Do privacy considerations hamper cloud service adoption?

\end{enumerate}

\newpage

\begin{center}

  \emph{Conclusions}

\end{center}

\begin{enumerate}

\item Notebooks and Netbooks are comparable as cloud service consumption
  devices.

\item Notebooks and Netbooks are not comparable as local computing devices.

\item Cloud Services still have kinks the need ironing before mass adoption can
  be seen.

\item Broadband Internet connectivity, while not the issue in the tests
  conducted, is a factor in cloud service experience, especially with high data
  volume services like YouTube.

\item The ideal mobile platform would utilize cloud service exclusively so that
  power would only be needed to run the display and maintain Internet
  connectivity.  Thus, high-density batteries would last through the day with
  constant use.

\end{enumerate}

}

\end{document}
