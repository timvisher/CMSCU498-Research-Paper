My primary hypothesis is that mass adoption of ultra-mobile PCs will not happen
till customer hardware requirements are figured out.  Currently, the best
ultra-mobile PC battery life is 14 hours\footnote{Retrieved January 31, 2010
from \url{http://www.netbookreviews.net/asus/eee-pc-1005pe-review/}} from the
Eee PC 1005PE made by Netbook specialists ASUS.  That is not long enough to
allow it to be on all day without charging.  Without hardware providing both the
performance necessary to run today's cloud applications and the inter-charge
life expectancy to rival todays smart phones, Netbooks will remain a niche
product unless their price alone can be made compelling enough that people
accept the performance hit and begin to use them as their sole PC.

In close association with this hypothesis, the author intends to address the
following question: What would an ideal mobile offering for the average consumer
look like?  If it is indeed true that consumers still prefer traditional PCs
over Netbooks and Smart-Phones, what would get people to switch to more of a
mobile platform?  Would it be longer battery life?  Higher density screens?
Better software performance?  Would it matter if that performance was offered
via that cloud or would it have to be local?  These questions are especially
pertinent to the intended outcome of the research project.  The author intends
to create a paper prototype of a mobile platform that is ideal to meet average
consumer needs.

A second hypothesis is that Netbooks and traditional Notebook PCs are comparable
in the use of cloud services while they are unfavorably comparable in their
performance for traditional desktop usage.  In other words, when using a cloud
service, Notebooks and Netbooks will be just as performant, while doing the same
operations locally will produce a marked difference in performance.  This could
be proven by testing various cloud services and doing the same operations
locally.  A video could be encoded on a Notebook, a Netbook, and through both on
a service such as Vimeo or YouTube.  Time to completion could be
tested. Similarly, a large spreadsheet could be constructed on the 2 devices and
then on Google Docs, and performance could be compared.
