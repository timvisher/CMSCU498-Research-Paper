\section{Questions and Hypotheses}

The researcher's primary hypothesis is that mass adoption of ultra-mobile PCs
has not occurred of yet because they have not been optimized for Cloud Service
usage which makes them less powerful and cheaper Notebooks.  Currently,
the best ultra-mobile PC battery life is 14 hours\footnote{Retrieved January 31, 2010      % Can't proove this hypothesis... But it's not a hypothesis!
from \url{http://www.netbookreviews.net/asus/eee-pc-1005pe-review/}} from the
Eee PC 1005PE made by Netbook specialists ASUS.  That is not long enough to
allow it to be on all day without charging.  These computers miss the
opportunity to take advantage of Cloud Services by failing to implement a
hardware redesign that focusses on providing solid access to Cloud Services via
a platform with a laptop-like form factor but much higher battery life.  Doing
so could have Netbooks rival smart phones regarding life expectancy and usability.

%% In close association with this hypothesis, the author intends to address the
%% following question: What would an ideal mobile offering for the average consumer
%% look like?  If it is indeed true that consumers still prefer traditional PCs
%% over Netbooks and Smart-Phones, what would get people to switch to more of a
%% mobile platform?  Would it be longer battery life?  Higher density screens?
%% Better software performance?  Would it matter if that performance was offered
%% via that cloud or would it have to be local?  These questions are especially
%% pertinent to the intended outcome of the research project.  The author intends
%% to create a paper prototype of a mobile platform that is ideal to meet average
%% consumer needs.

A second hypothesis is that Netbooks and traditional Notebook PCs are
comparable % Why would I delete this?
in the use of cloud services while they are unfavorably comparable in their
performance for traditional desktop usage.  In other words, when using a cloud
service, Notebooks and Netbooks will be just as performant, while doing the same
operations locally will produce a marked difference in performance.
