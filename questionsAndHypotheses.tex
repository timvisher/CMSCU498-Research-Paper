\documentclass[12pt,oneside,letterpaper]{article}
\author{Tim Visher\\ CMSCU 498\\ Chestnut Hill College}
\title{Questions and Hypotheses}
\date{February 1, 2010}

\usepackage[colorlinks=true]{hyperref}

\begin{document}

\maketitle

\begin{enumerate}

\item My primary hypothesis is that mass adoption of ultra-mobile PCs will not
  happen till customer hardware requirements are figured out.  Currently, the
  best ultra-mobile PC battery life is 14 hours\footnote{Retrieved January 31,
    2010 from\\ \url{http://www.netbookreviews.net/asus/eee-pc-1005pe-review/}}
  %TODO Enter into bibliography
  from the Eee PC 1005PE made by Netbook specialists ASUS.  That is not long
  enough to allow it to be on all day without charging.  Without hardware
  providing both the performance necessary to run today's cloud applications and
  the inter-charge life expectancy to rival todays smart phones, Netbooks will
  remain a niche product unless their price alone can be made compelling enough
  that people accept the performance hit and begin to use them as their sole PC.

  In close association with this hypothesis, the author intends to address the
  following questions:

  \begin{enumerate}

  \item What about current mobile offerings is dissatisfying to the average
    consumer?  Despite enormous inroads into mobile PC sales made by Netbook
    manufacturers like ASUS and Acer, the primary drive for these PCs seems to
    be cost at this point\footnote{Where did this come from?}. %TODO Update
                                                               %reference

  \item What would an ideal mobile offering for the average consumer look like?
    If it is indeed true that consumers still prefer traditional PCs over
    Netbooks and Smart-Phones, what would get people to switch to more of a
    mobile platform?  Would it be longer battery life?  Higher density screens?
    Better software performance?  Would it matter if that performance was
    offered via that cloud or would it have to be local?

  \end{enumerate}

  Both of these questions are especially pertinent to the intended outcome of
  the research project.  The author intends to create a paper prototype of a
  mobile platform that is ideal to meet average consumer needs.

\item A second hypothesis is that Netbooks and traditional Notebook PCs are
  comparable in the use of cloud services while they are unfavorably comparable
  in their performance for traditional desktop usage.  In other words, when
  using a cloud service, Notebooks and Netbooks will be just as performant,
  while doing the same operations locally will produce a marked difference in
  performance.  This could be proven by testing various cloud services and doing
  the same operations locally.  A video could be encoded an a Notebook, a
  Netbook, and through both on a service such as Vimeo or YouTube.  Time to
  completion could be tested. Similarly, a large spreadsheet could be
  constructed on the 2 devices and then on Google Docs, and performance could be
  compared.

\item The author's third hypothesis is that many people believe they use more of
  their computer than they actually do.  At this point the author would have to
  measure in some way what demands normal uses makes on a CPU and how that
  performance differs between Netbooks and Notebooks.  The author could use the
  System Requirements from a variety of popular software, for instance, to get
  general benchmarks for what Systems are required to provide hardware-wise at
  this point.  The essential argument from many Netbook makers is that if you're
  primary use of the computer is browsing the Web, writing and receiving e-mail,
  and preparing documents and presentations in an office suite, then a Netbook
  will provide all the computing performance you need.  The author would like to
  investigate the validity of this point.

\item One question that the author would like to answer during this project is
  whether considerations surrounding privacy affect peoples' opinion of cloud
  services at this point or whether most people comfortable with a corporation
  housing and protecting their private data.  The author imagines that most
  people who grew up on the Internet would have a lack of a sense of privacy and
  so would not mind giving that data away to be held by someone else that they
  have no direct control over.  This would seem to be born out by the heavy use
  of services like Facebook to store much of our private information.  The
  author would also expect the answers to this question to be stratified across
  generations, with the older generations being more likely to not trust a
  service with any of their data.

\end{enumerate}

Almost certainly, the last 2 points will be impossible to get to in the time the
author has to complete the research.  They are, nonetheless, interesting thoughts.

\end{document}
